\newpage
\section{Ejercicio 1}

%%%%%%%%%%%%%%%%%%%%%%%%%%%%%%%%%%%%%%%%%%%%%%%%%%%%%%%%%%%%%%%%%%%%%%
% Introduccion
%%%%%%%%%%%%%%%%%%%%%%%%%%%%%%%%%%%%%%%%%%%%%%%%%%%%%%%%%%%%%%%%%%%%%%
\subsection{Introducci'on}
Para comenzar a pensar este ejercicio, se descart'o de lleno la opci'on de fuerza bruta. Se busc'o una que podara la mayor'ia de los casos que no sirven. Donde se ahorran la mayor'ia de los casos es al encontrar un caso que no sirve, se siga buscando en ese mismo conjunto, un elemento m'as abajo.

Lo que se hizo en este algoritmo es primero colocar los primeros elementos de ambos conjuntos, lo cual esta garantizado que es la tupla m'as grande en valor. Luego, se elige arbitrariamente el primero de uno, con el segundo del otro. Se buscan los mayores a esta tupla, descartando al encontrar un caso menor, ya que si, por ejemplo A[i]+B[j] es menor al valor de referencia, A[i]+B[j+1] tambi'en lo ser'a, por la precondici'on de que ambos conjuntos vienen ordenados. Luego de conseguir 'estas sumas, se vuelve a buscar mayores, pero en' este caso con los pr'oximos dos elementos del m'ismo 'indice. 'Esta vez se busca tanto las variaciones del conjunto B con el A, como las del A con el B. Luego, quedan generadas 3 listas con candidatos para ingresar. Se ordena 'esta lista y se agregan a la lista final hasta que se llene. En caso contrario, que falten casos, se vuelve a implementar el procedimiento.

Para ordenar la lista, primero se tienen que ordenar las listas candidatas. Para lo primero se utiliz'o un Quick Sort por sobre el Selection Sort, ya que, aunque ambos tienen la misma complejidad en peor caso, el Quick Sort se comporta mucho mejor en casos promedio, como se puede ver en el gr'afico de los resultados. Para el segundo, se utiliza un Merge Sort para crear una nueva lista ordenada desde las tres listas anteriores.

%%%%%%%%%%%%%%%%%%%%%%%%%%%%%%%%%%%%%%%%%%%%%%%%%%%%%%%%%%%%%%%%%%%%%%
% Detalles de implementacion
%%%%%%%%%%%%%%%%%%%%%%%%%%%%%%%%%%%%%%%%%%%%%%%%%%%%%%%%%%%%%%%%%%%%%%
\subsection{Detalles de implementaci'on}
Se cre'o la clase Datos para contener a los dos conjuntos iniciales y se cre'o la clase Instancia para contener ambos conjuntos y el n del tama'no.

Como detalle, se puede mencionar que el algoritmo se optimiz'o un poco m'as utilizando en algunos casos Linked Lists en vez de Array Lists, ya que las primeras tienen una complejidad de lectura constante y de inserci'on lineal. 'Esto ahorr'o la lectura de los conjuntos iniciales, que se realiza en varias oportunidades. Los conjuntos, tanto temporales como el del resultado, siguen siendo ArrayLists, ya que en ellos se inserta m'as de lo que se lee, y en este caso, la inserci'on es constante.

El resto del algoritmo se realiz'o como se detalla en la introducci'on.


%%%%%%%%%%%%%%%%%%%%%%%%%%%%%%%%%%%%%%%%%%%%%%%%%%%%%%%%%%%%%%%%%%%%%%
% Pseudocodigos
%%%%%%%%%%%%%%%%%%%%%%%%%%%%%%%%%%%%%%%%%%%%%%%%%%%%%%%%%%%%%%%%%%%%%%
\clearpage
\subsection{Pseudoc'odigos}

\algoritmo{Mayores($A,B$)}{Devuelve mayores sumas entre $A$ y $B$}{O($n^2$)}
\begin{algorithmic}[1]
\REQUIRE A y B ordenados
\STATE $i \leftarrow 0$
\WHILE{n $>$ 0}
	\STATE $ret.agregar(A[i],B[i])$ 
	\STATE $valor \leftarrow A[i]+B[i+1]$
	\STATE $mayores \leftarrow BuscoMayor(valor,A,B,i)$
	\STATE $valor1 \leftarrow A[i+1]+B[i+1]$
	\STATE $menores \leftarrow BuscoMayor(valor1,A,B,i)$
	\STATE $menores1 \leftarrow BuscoMayor(valor1,B,A,i)$
	\STATE $Ordenar(mayores)$
	\STATE $Ordenar(menores)$
	\STATE $Ordenar(menores1)$
	\STATE $Ingresar.agregar(mayores,menores,menores1)$
	\STATE $OrdenarListas(ingresar)$
	\STATE $j \leftarrow 0$
	\WHILE{n $>$ 0}
		\STATE $ret.agregar(ingresar(j))$
		\STATE $j++$
	\ENDWHILE
	\STATE $i++$
\ENDWHILE

\end{algorithmic}

\vspace{2em}

\algoritmo{BuscoMayor(valor,A,B,i)}{Devuelve la lista de tuplas mayores a $valor$}{O($n^2$)}
\begin{algorithmic}[1]
\REQUIRE A y B ordenados
\STATE $k \leftarrow 0$
\WHILE{k $\leq$ i}
	\STATE $j = i+1$
	\WHILE{j $<$ tama'no(A)}
		\IF{A[j]+B[k] $\geq$ valor}
			\STATE $ret.agregar(A[j],B[k])$
			\STATE $j++$
		\ELSE
			\STATE $break$
		\ENDIF
	\ENDWHILE
	\STATE $k++$
\ENDWHILE
\RETURN ret

\end{algorithmic}

\vspace{2em}

Los Algoritmos Ordenar y OrdenarListas corresponden a los habituales algoritmos Quick Sort y Merge Sort, cambiando los enteros por la lista de tuplas. Sus complejidades respectivas en peor caso son $x^2$ y $x*log(x)$ respectivamente.


%%%%%%%%%%%%%%%%%%%%%%%%%%%%%%%%%%%%%%%%%%%%%%%%%%%%%%%%%%%%%%%%%%%%%%
% Analisis de complejidad
%%%%%%%%%%%%%%%%%%%%%%%%%%%%%%%%%%%%%%%%%%%%%%%%%%%%%%%%%%%%%%%%%%%%%%
\subsection{Analisis de complejidad}
{\bf\small Modelo Uniforme}
\subsubsection{Mayores}

Tama'no de la entrada = 2 * n = x con n cantidad de elementos del conjunto.

La complejidad de Mayores es de O($4n^2+n+n*log(n)$). Los $n^2$ corresponden a las 3 veces que llama al algoritmo BuscoMayor, la $n*log(n)$ corresponde al Merge Sort para ordenar a los candidatos y la $n$ restante es del agregado de los candidatos a la lista de retorno Ret.

El resto de las operaciones son basicas, o, en el caso del costo del acceso a las listas, las cuales son Linked Lists, que es constante.

En funci�n del tama'no de la entrada, la complejidad quedaria O($x^2$).

\subsubsection{BuscoMayor}

Este algoritmo recorre $i$ veces la lista B desde $n-i$ hasta $n$. En el peor caso es que $i$ sea $n/2$, lo cual hace recorrer $n/2 * n/2$ veces.

Por lo tanto, la complejidad del algoritmo es O($n^2$). Las restantes operaciones realizadas son O(1) dentro del algoritmo.

En funci'on del tama'no de la entrada, la complejidad queda cuadratica, O($x^2$).


%%%%%%%%%%%%%%%%%%%%%%%%%%%%%%%%%%%%%%%%%%%%%%%%%%%%%%%%%%%%%%%%%%%%%%
% Resultados
%%%%%%%%%%%%%%%%%%%%%%%%%%%%%%%%%%%%%%%%%%%%%%%%%%%%%%%%%%%%%%%%%%%%%%
\clearpage
\subsection{Resultados}
\imagen{img/Tp1Ej1(totales).png}{14}{Cantidad de operaciones b'asicas del algoritmo completo}
\imagen{img/Tp1Ej1(val).png}{14}{Cantidad de operaciones b'asicas del algoritmo Valor}
\imagen{img/Tp1Ej1(ord).png}{14}{Cantidad de operaciones b'asicas del algoritmo Ordenar}
\imagen{img/Tp1Ej1(orl).png}{14}{Cantidad de operaciones b'asicas del algoritmo OrdenarListas}
\imagen{img/Tp1Ej1(may).png}{14}{Cantidad de operaciones b'asicas del algoritmo Mayores}
\imagen{img/Tp1Ej1(bus).png}{14}{Cantidad de operaciones b'asicas del algoritmo BuscoMayor}
\imagen{img/Tp1Ej1(QS vs SS).png}{14}{Comparaci'on de Cantidad de operaciones b'asicas del algoritmo Quick Sort contra Selection Sort}
\imagen{img/Tp1Ej1(ah vs bus).png}{14}{Cantidad de operaciones b'asicas ahorradas contra hechas}
\clearpage

%%%%%%%%%%%%%%%%%%%%%%%%%%%%%%%%%%%%%%%%%%%%%%%%%%%%%%%%%%%%%%%%%%%%%%
% Comentarios
%%%%%%%%%%%%%%%%%%%%%%%%%%%%%%%%%%%%%%%%%%%%%%%%%%%%%%%%%%%%%%%%%%%%%%
\subsection{Discusi'on}
De las figuras se pueden sacar las siguientes conclusiones:

En la primera se ve que la complejidad te'orica es mayor a la pr'actica. 'Esto se debe a implementaciones internas de Java, las cuales amortizan muchos de los algoritmos de listas, haciendo menores la cantidad de operaciones.
Desde la segunda a la sexta, se ven las cantidades de operaciones desglosadas en cada algoritmo en particular. Observar que la del algoritmo principal es pr'acticamente lineal en funcion de la cantidad de elementos del conjunto. 
En el s'eptimo se ve la comparacion entre Quick sort y Selection sort, y se nota la clara diferencia entre ambos. Si bien los dos tienen igual complejidad en peor caso, en caso del Quick Sort es $n*log(n)$ en caso promedio.
En el octavo se ve claramente los ahorros de iteraciones en BuscoMayor al aplicar el break en el else del if. Esto ahorra much'isimas iteraciones innecesarias.
