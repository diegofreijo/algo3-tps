\documentclass[spanish, a4paper, 11pt]{article}

\usepackage[a4paper,margin=3.5cm,top=3.0cm,bottom=3.0cm]{geometry}	% Define los margenes
\usepackage[spanish,activeacute]{babel}								% Idioma castellano
\usepackage{caratula}														% Caratula de Algo2
%\usepackage[a4paper=true,pagebackref=true]{hyperref}				% Agrega la TOC al PDF e hipervinculos
%\usepackage[pdftex]{graphicx} 											% Permite insertar graficos
\usepackage{fancyhdr}														% Permite manejo de cabeceras de pagina
\usepackage{eufrak}															% Usado en el enunciado del trabajo
\usepackage{latexsym}
\usepackage{algorithmic}													% Para escribir los algos
\usepackage{dsfont}															% Para el simbolo de naturales


% Estilo de pagina para tener las cabeceras
\pagestyle{fancy}
\lhead{AyED III - TP1}
\rhead{Matias Blanco - Diego Freijo - Maximiliano Giusto}

% Numeracion de paginas
\pagenumbering{arabic}
\parskip=1.5ex

% Seteo de estilo de los algoritmos
\algsetup{indent=2em}

\newcommand{\imagen}[3]
{
	\begin{figure}[h]
	  \centering
%	    \includegraphics[width=#2cm]{#1}
	  \caption{#3}
	\end{figure}
}

\newcommand{\nat}{\mathds{N}}
\newcommand{\algoritmo}[3]{\noindent {\bf\underline{#1}:} #2 $\longrightarrow$ #3}
\newcommand{\superindice}[1]{$^\textrm{{\tiny #1}}$}
\newcommand{\subsubsubsection}[1]{

{\bf\small #1}

}
\newcommand{\negrita}[1]{{\bf #1}}

%%%%%%%%%%%%%%%%%%%%%%%%%%%%%%%%%%%%%%%%%%%%%%%%%%%%%%%%%%%%%%
%%%%%%%%%%%%%%%%%%%%%%%%%%%%%%%%%%%%%%%%%%%%%%%%%%%%%%%%%%%%%%%%%%%%%%%%%%%%%%%%%%%%
%%%%%   Inicio del documento
%%%%%%%%%%%%%%%%%%%%%%%%%%%%%%%%%%%%%%%%%%%%%%%%%%%%%%%%%%%%%%%%%%%%%%%%%%%%%%%%%%%%
%%%%%%%%%%%%%%%%%%%%%%%%%%%%%%%%%%%%%%%%%%%%%%%%%%%%%%%%%%%%%%

\begin{document}

%%%%%%%%%%%%%%%%%%%%%%%%%%%%%%%%%%%%%%%%%%%%%%%%%%%%%%%%%%%%%%%%%%%%%%
% Caratula
%%%%%%%%%%%%%%%%%%%%%%%%%%%%%%%%%%%%%%%%%%%%%%%%%%%%%%%%%%%%%%%%%%%%%%
\materia{Algoritmos y Estructuras de Datos III}
\submateria{Primer Cuatrimestre de 2007}
\titulo{Trabajo Pr'actico 1}
\subtitulo{}
\integrante{Blanco, Matias}{508/05}{matiasblanco18@gmail.com}
\integrante{Freijo, Diego}{4/05}{giga.freijo@gmail.com}
\integrante{Giusto, Maximiliano}{486/05}{maxi.giusto@gmail.com}
\maketitle

\subsection*{Palabras Clave}
Complejidad algor'itmica, modelo uniforme, modelo logar'itmico.


%%%%%%%%%%%%%%%%%%%%%%%%%%%%%%%%%%%%%%%%%%%%%%%%%%%%%%%%%%%%%%%%%%%%%%
% Indice
%%%%%%%%%%%%%%%%%%%%%%%%%%%%%%%%%%%%%%%%%%%%%%%%%%%%%%%%%%%%%%%%%%%%%%
\clearpage
\tableofcontents
\clearpage

%%%%%%%%%%%%%%%%%%%%%%%%%%%%%%%%%%%%%%%%%%%%%%%%%%%%%%%%%%%%%%%%%%%%%%
% Ejercicios
%%%%%%%%%%%%%%%%%%%%%%%%%%%%%%%%%%%%%%%%%%%%%%%%%%%%%%%%%%%%%%%%%%%%%%
\newpage
\section{Ejercicio 1}

%%%%%%%%%%%%%%%%%%%%%%%%%%%%%%%%%%%%%%%%%%%%%%%%%%%%%%%%%%%%%%%%%%%%%%
% Introduccion
%%%%%%%%%%%%%%%%%%%%%%%%%%%%%%%%%%%%%%%%%%%%%%%%%%%%%%%%%%%%%%%%%%%%%%
\subsection{Introducci'on}
Para comenzar a pensar este ejercicio, se descart'o de lleno la opci'on de fuerza bruta. Se busc'o una que podara la mayor'ia de los casos que no sirven. Donde se ahorran la mayor'ia de los casos es al encontrar un caso que no sirve, se siga buscando en ese mismo conjunto, un elemento m'as abajo.

Lo que se hizo en este algoritmo es primero colocar los primeros elementos de ambos conjuntos, lo cual esta garantizado que es la tupla m'as grande en valor. Luego, se elige arbitrariamente el primero de uno, con el segundo del otro. Se buscan los mayores a esta tupla, descartando al encontrar un caso menor, ya que si, por ejemplo A[i]+B[j] es menor al valor de referencia, A[i]+B[j+1] tambi'en lo ser'a, por la precondici'on de que ambos conjuntos vienen ordenados. Luego de conseguir 'estas sumas, se vuelve a buscar mayores, pero en' este caso con los pr'oximos dos elementos del m'ismo 'indice. 'Esta vez se busca tanto las variaciones del conjunto B con el A, como las del A con el B. Luego, quedan generadas 3 listas con candidatos para ingresar. Se ordena 'esta lista y se agregan a la lista final hasta que se llene. En caso contrario, que falten casos, se vuelve a implementar el procedimiento.

Para ordenar la lista, primero se tienen que ordenar las listas candidatas. Para lo primero se utiliz'o un Quick Sort por sobre el Selection Sort, ya que, aunque ambos tienen la misma complejidad en peor caso, el Quick Sort se comporta mucho mejor en casos promedio, como se puede ver en el gr'afico de los resultados. Para el segundo, se utiliza un Merge Sort para crear una nueva lista ordenada desde las tres listas anteriores.

%%%%%%%%%%%%%%%%%%%%%%%%%%%%%%%%%%%%%%%%%%%%%%%%%%%%%%%%%%%%%%%%%%%%%%
% Detalles de implementacion
%%%%%%%%%%%%%%%%%%%%%%%%%%%%%%%%%%%%%%%%%%%%%%%%%%%%%%%%%%%%%%%%%%%%%%
\subsection{Detalles de implementaci'on}
Se cre'o la clase Datos para contener a los dos conjuntos iniciales y se cre'o la clase Instancia para contener ambos conjuntos y el n del tama'no.

Como detalle, se puede mencionar que el algoritmo se optimiz'o un poco m'as utilizando en algunos casos Linked Lists en vez de Array Lists, ya que las primeras tienen una complejidad de lectura constante y de inserci'on lineal. 'Esto ahorr'o la lectura de los conjuntos iniciales, que se realiza en varias oportunidades. Los conjuntos, tanto temporales como el del resultado, siguen siendo ArrayLists, ya que en ellos se inserta m'as de lo que se lee, y en este caso, la inserci'on es constante.

El resto del algoritmo se realiz'o como se detalla en la introducci'on.


%%%%%%%%%%%%%%%%%%%%%%%%%%%%%%%%%%%%%%%%%%%%%%%%%%%%%%%%%%%%%%%%%%%%%%
% Pseudocodigos
%%%%%%%%%%%%%%%%%%%%%%%%%%%%%%%%%%%%%%%%%%%%%%%%%%%%%%%%%%%%%%%%%%%%%%
\clearpage
\subsection{Pseudoc'odigos}

\algoritmo{Mayores($A,B$)}{Devuelve mayores sumas entre $A$ y $B$}{O($n^2$)}
\begin{algorithmic}[1]
\REQUIRE A y B ordenados
\STATE $i \leftarrow 0$
\WHILE{n $>$ 0}
	\STATE $ret.agregar(A[i],B[i])$ 
	\STATE $valor \leftarrow A[i]+B[i+1]$
	\STATE $mayores \leftarrow BuscoMayor(valor,A,B,i)$
	\STATE $valor1 \leftarrow A[i+1]+B[i+1]$
	\STATE $menores \leftarrow BuscoMayor(valor1,A,B,i)$
	\STATE $menores1 \leftarrow BuscoMayor(valor1,B,A,i)$
	\STATE $Ordenar(mayores)$
	\STATE $Ordenar(menores)$
	\STATE $Ordenar(menores1)$
	\STATE $Ingresar.agregar(mayores,menores,menores1)$
	\STATE $OrdenarListas(ingresar)$
	\STATE $j \leftarrow 0$
	\WHILE{n $>$ 0}
		\STATE $ret.agregar(ingresar(j))$
		\STATE $j++$
	\ENDWHILE
	\STATE $i++$
\ENDWHILE

\end{algorithmic}

\vspace{2em}

\algoritmo{BuscoMayor(valor,A,B,i)}{Devuelve la lista de tuplas mayores a $valor$}{O($n^2$)}
\begin{algorithmic}[1]
\REQUIRE A y B ordenados
\STATE $k \leftarrow 0$
\WHILE{k $\leq$ i}
	\STATE $j = i+1$
	\WHILE{j $<$ tama'no(A)}
		\IF{A[j]+B[k] $\geq$ valor}
			\STATE $ret.agregar(A[j],B[k])$
			\STATE $j++$
		\ELSE
			\STATE $break$
		\ENDIF
	\ENDWHILE
	\STATE $k++$
\ENDWHILE
\RETURN ret

\end{algorithmic}

\vspace{2em}

Los Algoritmos Ordenar y OrdenarListas corresponden a los habituales algoritmos Quick Sort y Merge Sort, cambiando los enteros por la lista de tuplas. Sus complejidades respectivas en peor caso son $x^2$ y $x*log(x)$ respectivamente.


%%%%%%%%%%%%%%%%%%%%%%%%%%%%%%%%%%%%%%%%%%%%%%%%%%%%%%%%%%%%%%%%%%%%%%
% Analisis de complejidad
%%%%%%%%%%%%%%%%%%%%%%%%%%%%%%%%%%%%%%%%%%%%%%%%%%%%%%%%%%%%%%%%%%%%%%
\subsection{Analisis de complejidad}
{\bf\small Modelo Uniforme}
\subsubsection{Mayores}

Tama'no de la entrada = 2 * n = x con n cantidad de elementos del conjunto.

La complejidad de Mayores es de O($4n^2+n+n*log(n)$). Los $n^2$ corresponden a las 3 veces que llama al algoritmo BuscoMayor, la $n*log(n)$ corresponde al Merge Sort para ordenar a los candidatos y la $n$ restante es del agregado de los candidatos a la lista de retorno Ret.

El resto de las operaciones son basicas, o, en el caso del costo del acceso a las listas, las cuales son Linked Lists, que es constante.

En funci�n del tama'no de la entrada, la complejidad quedaria O($x^2$).

\subsubsection{BuscoMayor}

Este algoritmo recorre $i$ veces la lista B desde $n-i$ hasta $n$. En el peor caso es que $i$ sea $n/2$, lo cual hace recorrer $n/2 * n/2$ veces.

Por lo tanto, la complejidad del algoritmo es O($n^2$). Las restantes operaciones realizadas son O(1) dentro del algoritmo.

En funci'on del tama'no de la entrada, la complejidad queda cuadratica, O($x^2$).


%%%%%%%%%%%%%%%%%%%%%%%%%%%%%%%%%%%%%%%%%%%%%%%%%%%%%%%%%%%%%%%%%%%%%%
% Resultados
%%%%%%%%%%%%%%%%%%%%%%%%%%%%%%%%%%%%%%%%%%%%%%%%%%%%%%%%%%%%%%%%%%%%%%
\clearpage
\subsection{Resultados}
\imagen{img/Tp1Ej1(totales).png}{14}{Cantidad de operaciones b'asicas del algoritmo completo}
\imagen{img/Tp1Ej1(val).png}{14}{Cantidad de operaciones b'asicas del algoritmo Valor}
\imagen{img/Tp1Ej1(ord).png}{14}{Cantidad de operaciones b'asicas del algoritmo Ordenar}
\imagen{img/Tp1Ej1(orl).png}{14}{Cantidad de operaciones b'asicas del algoritmo OrdenarListas}
\imagen{img/Tp1Ej1(may).png}{14}{Cantidad de operaciones b'asicas del algoritmo Mayores}
\imagen{img/Tp1Ej1(bus).png}{14}{Cantidad de operaciones b'asicas del algoritmo BuscoMayor}
\imagen{img/Tp1Ej1(qs_vs_ss).png}{14}{Comparaci'on de Cantidad de operaciones b'asicas del algoritmo Quick Sort contra Selection Sort}
\imagen{img/Tp1Ej1(ah_vs_bus).png}{14}{Cantidad de operaciones b'asicas ahorradas contra hechas}
\clearpage

%%%%%%%%%%%%%%%%%%%%%%%%%%%%%%%%%%%%%%%%%%%%%%%%%%%%%%%%%%%%%%%%%%%%%%
% Comentarios
%%%%%%%%%%%%%%%%%%%%%%%%%%%%%%%%%%%%%%%%%%%%%%%%%%%%%%%%%%%%%%%%%%%%%%
\subsection{Discusi'on}
De las figuras se pueden sacar las siguientes conclusiones:

En la primera se ve que la complejidad te'orica es mayor a la pr'actica. 'Esto se debe a implementaciones internas de Java, las cuales amortizan muchos de los algoritmos de listas, haciendo menores la cantidad de operaciones.
Desde la segunda a la sexta, se ven las cantidades de operaciones desglosadas en cada algoritmo en particular. Observar que la del algoritmo principal es pr'acticamente lineal en funcion de la cantidad de elementos del conjunto. 
En el s'eptimo se ve la comparacion entre Quick sort y Selection sort, y se nota la clara diferencia entre ambos. Si bien los dos tienen igual complejidad en peor caso, en caso del Quick Sort es $n*log(n)$ en caso promedio.
En el octavo se ve claramente los ahorros de iteraciones en BuscoMayor al aplicar el break en el else del if. Esto ahorra much'isimas iteraciones innecesarias.

 \clearpage

\newpage
\section{Ejercicio 2}

%%%%%%%%%%%%%%%%%%%%%%%%%%%%%%%%%%%%%%%%%%%%%%%%%%%%%%%%%%%%%%%%%%%%%%
% Introduccion
%%%%%%%%%%%%%%%%%%%%%%%%%%%%%%%%%%%%%%%%%%%%%%%%%%%%%%%%%%%%%%%%%%%%%%
\subsection{Introducci'on}
El primer algoritmo pensado para resolver este problema fue el de fuerza bruta, el cual recorre los todos los nodos de un 'arbol de busqueda y calcula la altura cada vez que llega a una hoja.

Dicho 'arbol empesar'ia con un nodo con un cero y cumple con la propiedad de que cada nodo (excepto las hojas) tiene una cantidad de hijos equivalente a: $$Cantidad\ Integrantes\ Familia - Valor\ Nodo - \# (enemistados\ con\ el)$$
Los valores que toman los hijos van desde $Valor\ Nodo\ +\ 1$ hasta $Cantidad\ Integrantes\ Familia$ sin tener en cuenta los valores enemistados con el nodo padre.

La cantidad de elementos, suponiendo que no hay enemistades en la familia, que tendria el 'arbol es $2^n\ -\ 1$, donde $n$ es la cantidad de integrantes de la familia.

Luego de tener el algoritmo de fuerza bruta nos vimos con la necesidad de empezar a {\it podar}, debido a que se realizan muchas comparaciones innecesarias que aumentaban la complejidad.

La {\it poda} se realizo teniendo en cuenta el siguiente razonamiento: se van calculando alturas (cada una equivale a un tama'no de fiesta), si la m'axima de 'estas es mayor a la cantidad personas que me quedan para armar una fiesta nueva, no sigo y me quedo con dicha altura.



%%%%%%%%%%%%%%%%%%%%%%%%%%%%%%%%%%%%%%%%%%%%%%%%%%%%%%%%%%%%%%%%%%%%%%
% Detalles de implementacion
%%%%%%%%%%%%%%%%%%%%%%%%%%%%%%%%%%%%%%%%%%%%%%%%%%%%%%%%%%%%%%%%%%%%%%
\subsection{Detalles de implementaci'on}
Para guardar los datos de entrada se creo una clase llamada $Datos$, la cual contiene el valor de $n$, de $m$ y una lista con los pares de enemitades.

El algoritmo que resuelve el problema, $TimeForParty$, consta de dos partes:
\begin{itemize}
\item La primera refiere a la primer parte del algoritmo, la que calcula cuantos familiares pueden invitarse "de una"\ ya que no est'an enemistados con nadie.

Adem'as en esta parte tambi'en se arman una lista con las personas que tienen alguna enemistad y una matriz, las cuales se usan en la segunda parte. Si la matriz tiene un $true$ en la posici'on $i\ j$ es porque $i$ y $j$ est'an enemistados entre si. En cambio, si hay un $false$, no.

\item La segunda parte es la m'as costosa en complejidad algor'itmica. Es el algoritmo recursivo($MasQueridos$) que recorre el 'arbol (y realiza la poda correspondiente) buscando la fiesta m'as grande que se puede hacer con las personas enemistadas. Este resultado es, luego, sumado al obtenido en la parte 1.

\end{itemize}

Algunas decisiones que se tomaron con respecto a la entrada son:
\begin{itemize}
\item $n$ no puede ser menor que 2, si una persona sola conforma la familia entonces no tiene ningun problema para armar la fiesta ya que no puede estar enemistado con el mismo.
\item $m$ no puede ser cero, si no hay enemistades en la familia no necesita el algoritmo.
\end{itemize}


%%%%%%%%%%%%%%%%%%%%%%%%%%%%%%%%%%%%%%%%%%%%%%%%%%%%%%%%%%%%%%%%%%%%%%
% Pseudocodigo
%%%%%%%%%%%%%%%%%%%%%%%%%%%%%%%%%%%%%%%%%%%%%%%%%%%%%%%%%%%%%%%%%%%%%%
\subsection{Pseudoc'odigo}

\algoritmo{TimeForParty($dato$)}{$La\ maxima\ cantidad\ de\ personas\ que\ se\ quieren$} %{O($log_2(n)*k^3$)}
\begin{algorithmic}[1]
\REQUIRE $n > 2,\ m > 0$
\STATE $invitados \leftarrow 0$
\FOR{$i = 1 \dots dato.n - 1$}
	\IF{$EstaEnAlgunPar(i, dato.paresEnemistados)$)}
		\STATE Agregar(listaProblematicos,$i$)
	\ELSE
		\STATE $invitados \leftarrow invitados + 1$
	\ENDIF
\ENDFOR

\FOR{$j = 0 \dots dato.m$}
	\STATE $matriz[(paresEnemistados(j))(0)][(paresEnemistados(j))(1)] = true$
	\STATE $matriz[(paresEnemistados(j))(1)][(paresEnemistados(j))(0)] = true$
\ENDFOR

\STATE $temp \leftarrow listaProblematicos.tamano()$
\STATE $Lista\ PosQueridos$
\STATE $Lista\ ParaAlturas $
\STATE $mayor\leftarrow 0$
\STATE $mayorAnt \leftarrow 0$

\FOR{$int i = 0 \dots temp$}
	\IF{$(temp - i) > mayorAnt$}
		\STATE $mayor = MasQueridos(matriz, listaProblematicos,PosQueridos,paraAlturas, i);$
		\IF{$mayor > mayorAnt$}
			\STATE $mayorAnt = mayor$
		\ENDIF
		\ELSE
			\STATE $break$
	\ENDIF
\ENDFOR
\STATE $invitados \leftarrow invitados + mayorAnt$

\RETURN $invitados$

\end{algorithmic}

\vspace{2em}

\algoritmo{MasQueridos($matriz\ m,Lista\ lP,Lista\ PQ,Lista\ pA,nat\ i$)}{Calcula la maxima cantidad de familiares peliados que se quieren}%{O($k^3$)}
\begin{algorithmic}[1]
\STATE $tam = lP.tamano()$
\STATE $Agregar(PQ,lP(n))$
\FOR{$k = n + 1 \dots tam - 1$}
	\IF{$!TieneEnemigos(lP(k),PQ,m) \&\& (tam-(k)+ PQ.tamano()) > MaximoElemento(pA)$}
		\STATE $Lista PQ2 \leftarrow PQ$
		\STATE $MasQueridos(m, lP, PQ2, pA, k)$
	\ENDIF
\ENDFOR
\STATE $Agregar(pA,PQ.tamano())$
\STATE $PQ.vaciar()$

\STATE $ret \leftarrow MaximoElemento(pA)$

\RETURN $ret$;

\end{algorithmic}

Aclaraci'on sobre las funciones:
\begin{itemize}
\item $EstaEnAlgunPar(entero,\ listaDeParesEnemistados)$: devuelve $true$ en el caso en que el entero que entra como parametro este en alguna de las peleas contenida por la lista de enemistados.
\item $TieneEnemigos(entero,\ lista,\ matriz)$: devuelve $true$ en el caso en que el entero este enemistado con alg'un elemento de la lista.
\item $MaximoElemento(lista)$: devuelve el m'aximo elemento de la lista.
\end{itemize}
%%%%%%%%%%%%%%%%%%%%%%%%%%%%%%%%%%%%%%%%%%%%%%%%%%%%%%%%%%%%%%%%%%%%%%
% Respuestas
%%%%%%%%%%%%%%%%%%%%%%%%%%%%%%%%%%%%%%%%%%%%%%%%%%%%%%%%%%%%%%%%%%%%%%
%\subsection{Respuestas}
%????????????

%%%%%%%%%%%%%%%%%%%%%%%%%%%%%%%%%%%%%%%%%%%%%%%%%%%%%%%%%%%%%%%%%%%%%%
% Analisis de complejidad
%%%%%%%%%%%%%%%%%%%%%%%%%%%%%%%%%%%%%%%%%%%%%%%%%%%%%%%%%%%%%%%%%%%%%%
\subsection{Analisis de complejidad}
Si la entrada tiene pocas peleas habr'a pocas llamadas recursivas y poca cantidad de operaciones. Ahora si hay demasiadas peleas, por ejemplo si todos estan enemistados con todos, no habr'a llamadas recursivas y por consiguiente la complejidad estar'a dada por la primera parte del algoritmo.El peor caso, para el algoritmo que resuelve el ejercicio, es cuando todos los elementos est'an peleados y pero no todos contra todos sino de forma disjunta.

Como el an'alisis se har'a teniendo en cuenta el peor caso se va suponer que $m$ = $n/2$. %y que todas las peleas son disjuntas

El tama'no de entrada es $2 * m = 2* n/2 = n$

La complejidad de la primera parte del algoritmo es O($n^2$) debido a que recorro $n$ veces la lista de pares enemistados que tiene como tama'no $n/2$, m'as la complejidad de crear la matriz, que tambi'en es O($n^2$).

La complejidad de la segunda parte del algoritmo est'a dada por $$O(cantidad de repeticiones)*O(MasQueridos)$$Como la cantidad de repeticiones ser'a menor o igual a $n$, la complejidad ser'a menor o igual a $$O(n)*O(MasQueridos)$$

El algoritmo de fuerza bruta tiene complejidad O($n*2^n$) debido a que recorre todos los nodos del 'arbol y cada vez que llega a una hoja calcula la altura hasta ella. MasQueridos hace lo mismo a excepci'on que no recorre todo el 'arbol sino que usa la condici'on de $poda$. Igualmente dicha condici'on, si bien mejora un poco la complejidad para valores chicos de $n$, sigue siendo del mismo orden.

Finalmente la complejidad del algoritmo TimeForParty es O($n^2*2^n$).

%%%%%%%%%%%%%%%%%%%%%%%%%%%%%%%%%%%%%%%%%%%%%%%%%%%%%%%%%%%%%%%%%%%%%%
% Resultados
%%%%%%%%%%%%%%%%%%%%%%%%%%%%%%%%%%%%%%%%%%%%%%%%%%%%%%%%%%%%%%%%%%%%%%
\clearpage
\subsection{Resultados}
\imagen{img/Tp1Ej2.png}{14}{Cantidad de operaciones b'asicas del algoritmo completo}
%%%%%%%%%%%%%%%%%%%%%%%%%%%%%%%%%%%%%%%%%%%%%%%%%%%%%%%%%%%%%%%%%%%%%%
% Comentarios
%%%%%%%%%%%%%%%%%%%%%%%%%%%%%%%%%%%%%%%%%%%%%%%%%%%%%%%%%%%%%%%%%%%%%%
\clearpage
\subsection{Conclusiones}
La complejidad te'orica es mayor que la pr'actica, esto se debe a que al aplicar O() perdemos algunas constantes que emparejar'ian m'as las lineas del gr'afico y a que algunas implentaciones de Java (sobre todo las hechas sobres listas) tiene baja complejidad (por ejemplo en el acceso ordenado a una lista).
%%%%%%%%%%%%%%%%%%%%%%%%%%%%%%%%%%%%%%%%%%%%%%%%%%%%%%%%%%%%%%%%%%%%%%
% Referencias
%%%%%%%%%%%%%%%%%%%%%%%%%%%%%%%%%%%%%%%%%%%%%%%%%%%%%%%%%%%%%%%%%%%%%%
%\subsection{Referencias}
%\begin{itemize}
%    \item Referencias???
%\end{itemize}

\clearpage

\newpage
\section{Ejercicio 3}

%%%%%%%%%%%%%%%%%%%%%%%%%%%%%%%%%%%%%%%%%%%%%%%%%%%%%%%%%%%%%%%%%%%%%%
% Introduccion
%%%%%%%%%%%%%%%%%%%%%%%%%%%%%%%%%%%%%%%%%%%%%%%%%%%%%%%%%%%%%%%%%%%%%%
\subsection{Introducci'on}
Al comenzar a pensar la soluci'on de 'este ejercicio, partimos de un algortimo por fuerza bruta e intentamos mejorarlo. Cre'imos que el punto clave en donde se podr'ian ahorrar operaciones era en la multiplicaci'on de matrices. Luego de varios intentos e investigaci'ones por internet, hallamos que exist'ian algoritmos de multiplicaci'on de matrices cuadradas m'as r'apidos que el {\it standart} de orden c'ubico sobre las dimensiones de la matriz. Por ejemplo, el algoritmo de Strassen\superindice{\cite{strassen}} del orden de O($k^{2.807}$) y el de Coppersmith-�Winograd\superindice{\cite{cw}} del orden de O($k^{2.376}$). Pero ninguno de ellos fue utilizado. La desici'on provino de notar que los algoritmos eran dif'iciles de implementar y solamente para obtener una peque'na disminuci'on en la complejidad. Adem'as, los c'alculos de complejidad (especialmente en el modelo logar'itmico) se hubiesen complejizado demasiado. Evidentemente, el gran salto entre la fuerza bruta y un algoritmo m'as 'optimo no estaba all'i.

Entonces se pens'o en atacar el problema m'as desde arriba. Cada multiplicaci'on entre matrices es una operaci'on muy costosa, por lo que nos pareci'o interesante lograr disminu'ir el n'umero de 'estas al realizar la potenciaci'on. La primer idea que surgi'o fue de suponer al exponente como potencia de 2, por ejemplo 4. Por fuerza bruta, el algoritmo hubiese hecho
$$A = A*A*A*A$$
costando tres multiplicaciones (complejidad lineal). Pero 'esto se puede mejorar, ya que
$$A = (A^2)^2$$
con lo que la cantidad de multiplicaciones requeridas hubiesen sido una para multiplicar $A$ por s'i misma ($A^2$) y otra para multiplicar el resultado por s'i mismo ($(A^2)^2$), disminuyendo en una tercera parte la cantidad de multiplicaciones. En efecto, la complejidad resultante (contando solamente multiplicaciones) ser'ia logar'itmica. Pero hab'ia un serio problema de fondo: el exponente no ten'ia porque ser potencia de dos.

Con ello en mente, buscamos informaci'on sobre como aplicar la misma idea a cualquier exponente. Y nos encontramos con el algortimo de potenciaci'on binaria ({\it binary power})\superindice{\cite{binpow}}. 'Este utiliza la misma idea que ten'iamos pero salva los casos que no son potencia de dos tomando la representaci'on binaria del exponente. As'i es como funciona:

\begin{itemize}
	\item Suponer que quiero realizar $A^n$.
	\item Suponer que la representaci'on binaria de $n$ es $b_{k-1}...b_0$, donde cada $b_i$ es un d'igito binario.
	\item Claramente 
\begin{equation}
	\label{sumabin}
	n = \sum_{i=0}^{k-1} b_i  2^i
\end{equation}
o sea que para formar a $n$ debo sumar aquellas potencias de 2 en donde su d'igito binario de $n$ es 1.
	\item De 'esta forma es f'acil construirse el algoritmo. La matriz resultado comienza siendo la identidad (considerar que equivale a $A^0$). A su vez tengo otra matriz con exponentes de $A$ potencias de dos (comienza siendo $A^{2^0} = A$). Voy tomando los valores de $b_0$ en adelante y en donde halla un uno, multiplico la matriz resultado por la de las potencias (ya que la multiplicaci'on entre las matrices repercute como una suma entre los exponentes). Al final de cada iteraci'on elevo al cuadrado la matriz de las potencias de 2 para pasar a la siguiente. Repitiendo con cada $b_i$ es como voy transformando el exponente del resultado en $n$. Al terminar con $b_{k-1}$, 'este va a ser efectivamente $n$ (y la justificaci'on es la sumatoria expresada en (\ref{sumabin})).
\end{itemize}

'Este algoritmo es del orden de $2\log_2(n)$ multiplicaciones en el peor caso, con lo que nos quedamos bastante satisfechos y fue el utilizado.


%%%%%%%%%%%%%%%%%%%%%%%%%%%%%%%%%%%%%%%%%%%%%%%%%%%%%%%%%%%%%%%%%%%%%%
% Detalles de implementacion
%%%%%%%%%%%%%%%%%%%%%%%%%%%%%%%%%%%%%%%%%%%%%%%%%%%%%%%%%%%%%%%%%%%%%%
\subsection{Detalles de implementaci'on}
Para facilitar la implementaci'on lo primero que se hizo fue una clase Matriz, la cual representa una matriz cuadrada y posee solamente los m'etodos necesarios para resolver 'este ejercicio (principalmente, Multiplicar y Potenciar).

Al algoritmo Potenciar se le hicieron peque'nos cambios al implementarlo con el 'unico fin de hacerlo m'as eficiente:
\begin{itemize}
\item La representaci'on binaria de $n$ no se calcula antes de comenzar a multiplicar matrices, sino que se realizan las dos operaciones a la vez. Se utiliz'o el algoritmo {\it standart} para obtenerla.
\item Se agreg'o una guarda a la multiplicaci'on de $pot2$ para evitar que se realize una multiplicaci'on in'util cuando ya se alcanz'o el resultado. 'Esto suma O($\log_2(n)$) operaciones b'asicas a la complejidad final del algoritmo, pero evita sumar el O($k^3$) de la complejidad de una multiplicaci'on, por lo que en la pr'actica genera una notable optimizaci'on.
\end{itemize}



%%%%%%%%%%%%%%%%%%%%%%%%%%%%%%%%%%%%%%%%%%%%%%%%%%%%%%%%%%%%%%%%%%%%%%
% Pseudocodigos
%%%%%%%%%%%%%%%%%%%%%%%%%%%%%%%%%%%%%%%%%%%%%%%%%%%%%%%%%%%%%%%%%%%%%%
\clearpage
\subsection{Pseudoc'odigos}

\algoritmo{Potenciar($A,n$)}{Devuelve $A^n$}{O($log_2(n)k^3$)}
\begin{algorithmic}[1]
\REQUIRE $A \in \nat^{k \times k}$
\IF{$n = 0$}
	\RETURN $I \in \nat^{k \times k}$
\ENDIF
\STATE $ret \leftarrow I$
\STATE $pot2 \leftarrow A$
\WHILE{$n \ge 1$}
	\IF{$2\ |\ n$}
		\STATE $n \leftarrow n\ /\ 2$ 
	\ELSE
		\STATE $n \leftarrow (n-1)\ /\ 2$
		\STATE $ret \leftarrow$ Multiplicar($ret,pot2$)
	\ENDIF
	\STATE $pot2 \leftarrow$ Multiplicar($pot2,pot2$) 
\ENDWHILE
\RETURN $ret$
\end{algorithmic}

\vspace{2em}

\algoritmo{Multiplicar($A,B$)}{Multiplica la matriz $A$ por $B$}{O($k^3$)}
\begin{algorithmic}[1]
\REQUIRE $A,B \in \nat^{k \times k}$
	\STATE $ret \in \nat^{k \times k}$
	\FOR{$i = 1 \dots k$}
		\FOR{$j = 1 \dots k$}
			\STATE $ret_{ij} \leftarrow \sum_{t=1}^{k}{A_{it}B_{tj}}$
		\ENDFOR
	\ENDFOR
	\RETURN $ret$
\end{algorithmic}


\clearpage


%%%%%%%%%%%%%%%%%%%%%%%%%%%%%%%%%%%%%%%%%%%%%%%%%%%%%%%%%%%%%%%%%%%%%%
% Analisis de complejidad
%%%%%%%%%%%%%%%%%%%%%%%%%%%%%%%%%%%%%%%%%%%%%%%%%%%%%%%%%%%%%%%%%%%%%%
\subsection{An'alisis de complejidad}
\subsubsection{Multiplicar}

\subsubsubsection{Modelo Uniforme}

F'acilmente se puede ver en el algoritmo que para cada fila de $A$ (las cuales son $k$) y cada columna de $B$ (las cuales tambi'en son $k$) se debe recorrer cada elemento de ellas (los cuales, nuevamente, son $k$). Notar que siempre realiza lo mismo, sin importar de $k$ o los valores en la matriz, por lo que no hay mejor ni peor caso.
 
Luego, la complejidad uniforme es O($k^3$).

El tama'no de la entrada ser'a igual a la suma de los tama'nos de todos los elementos de cada matriz:
$$t = \sum_{i=1}^k\sum_{j=1}^k \log_2(A_{ij})+\log_2(B_{ij})$$

Pero como, para acotar, se considera que todos los elementos de cada una son el mayor de ambas (llamado $m$), el tama'no se puede expresar como
$$t = \sum_{i=1}^k\sum_{j=1}^k \log_2(m)+\log_2(m) = 2k^2\log_2(m) \in O(k^2\log_2(m))$$

Por lo tanto, la complejidad uniforme es O($\sqrt{t/2}\ t/2$) = O($t^{3/2}$), con lo que es \negrita{superlineal} en funci'on del tama'no de entrada.

\vspace{2em}

\subsubsubsection{Modelo Logar'itmico}

Para facilitar los c'alculos, se considerar'a que todos los valores de la matriz ser'an el m'as grande de todos y el que, por consiguiente, m'as espacio ocupe. 'Esto no afectar'a la validez del resultado ya que se estar'a calculando un orden, una cota superior. Supongo que $m$ es 'este valor. Luego, el espacio que ocupar'a su representaci'on binaria ser'a $\lceil \log_2(m+1) \rceil$. Como es del orden de O($\log_2(m)$), simplemente se considerar'a a 'este como el tama'no de $m$. De 'esta forma, realizar $m+m$ y $m*m$ se considerar'a que cuestan O($2\log_2(m)$) = O($\log_2(m)$) cada operaci'on.

En Multiplicar, para calcular cada valor del resultado se deben multiplicar $m$ con $m$ tantas veces como la dimension de la matriz. 'Estos productos deber'an ser sumados entre s'i. Primero $m^2+m^2$, luego $2m^2+m^2$, luego $3m^2+m^2$, etc. Es decir que cada suma costar'a 
$$ O(\log_2(cm^2)+\log_2(m^2)) = O(2\log_2(m) + \log_2(c) + 2\log_2(m)) = O(\log_2(m))$$

Por lo tanto, para calcular un valor de la matriz resultado, se deber'an calcular $k$ productos cada uno con un costo de O($\log_2(m)$) y $k-1$ sumas cada una con el mismo costo. 'Esto se deber'a realizar para cada elemento de la matriz resultado, la cual contiene $k^2$ de ellos. Luego, la complejidad logar'itmica de Multiplicar estar'a dada por
$$O((k\log_2(m) + (k-1)\log_2(m))k^2) = O(k^3\log_2(m))$$

Se verifica f'acilemnte que
$$t^2 \geq k^3\log_2(m) \Longleftrightarrow k^4\log^2_2(m) \geq k^3\log_2(m) \Longleftrightarrow k \geq 1$$

Por lo tanto, la complejidad logar'itmica se puede expresar como
$$O(t^2)$$
con lo que ser'a \negrita{cuadr'atica} en funci'on del tama'no de la entrada.


\subsubsection{Potenciar}

\subsubsubsection{Modelo Uniforme}

Dado que una multiplicaci'on entre matrices es muy costosa, y que es la m'as costosa en 'este algoritmo\footnote{Notar que los costos de copiar $A$ en $pot2$ e $I$ en $ret$ son del orden de O($k^2$) cada una, pero como s'olo se ejecutan una vez, y la complejidad de Multiplicar es mayor por ser O($k^3$), se desprecian.},  el peor de los casos para Potenciar ser'a cuando deba realizar la mayor cantidad de 'estas. Como $pot2$ siempre ser'a elevada al cuadrado, en cada iteraci'on se realizar'a por lo menos una multiplicaci'on. Pero la otra, correspondiente a $ret * pot2$, s'olo cuando el $n$ actual sea impar. 'Esto, como mejor se explic'o en la Introducci'on del presente, sucede si se encuentra un 1 en la representaci'on binaria de $n$. 

Por lo tanto, el peor $n$ de entrada ser'a aquel con todos unos en su representaci'on binaria. Y 'esto es equivalente a decir que
$$n = 2^r-1$$
para alg'un $r\in\nat$.

Ya que con un $n$ as'i se realizar'an 2 multiplicaciones por cada d'igito binario suyo (excepto en la 'ultima iteraci'on que realiza s'olo una), y que la cantidad de 'estos es $\log_2(n)$, la cantidad de multiplicaciones ser'an O($2*\log_2(n) - 1$). Pero, como cada multiplicaci'on es O($k^3$), la complejidad uniforme de Potenciar ser'a
$$O(2\log_2(n)k^3) = O(k^3\log_2(n))$$

Notar que el mejor caso ser'a aquel donde $n$ tenga la mayor cantidad de ceros posible en su representaci'on binaria. 'Es decir, $n = 100\dots 00$. All'i, s'olo realizar'ia $\log_2(n)$ multiplicaciones para $pot2$ y luego una m'as para hacer $ret*pot2$ (que en realidad ser'ia $I*A^n$). Por lo tanto, el mejor caso ser'a con $n$ potencia de dos.

El tama'no de la entrada ser'a igual a la suma del tama'no de todos los elementos de $A$ mas el de $n$. Como se considera que todos los elementos de $A$ son el m'aximo de ellos (llamado $m$), el tama'no de la entrada ser'a
$$t = \log_2(n) + \sum_{i=1}^k\sum_{j=1}^k \log_2(m) = k^2\log_2(m) + \log_2(n)$$

Tratando de acotar la complejidad, se puede ver primero que
$$t^2 = (k^2\log_2(m) + \log_2(n))^2 = $$
$$k^4\log^2_2(m) + 2k^2\log_2(m)\log_2(n) + \log^2_2(n) \geq k^2\log_2(n)$$
y que
$$\sqrt{t} = \sqrt{k^2\log_2(m) + \log_2(n)} \geq \sqrt{k^2} = k$$

Por lo tanto, la complejidad puede ser expresada como
$$O(t^2\sqrt{t}) = O(t^{5/2})$$
con lo que 'esta ser'ia \negrita{supercuadr'atica}\footnote{Nombre generado por inducci'on en \negrita{superlineal}. Tranquilamente se podr'ia afirmar que es \negrita{c'ubica}.} en funci'on del tama'no de la entrada.

\vspace{2em}

\subsubsubsection{Modelo Logar'itmico}
Dado el enunciado, se considerar'a que $n = 2^r$ para alg'un $r\in\nat$, adem'as de las suposiciones hechas en el c'alculo de Multiplicar.

Sea $m_1$ el mayor elemento de $A$. Como se considera que todos los elementos de la matriz son el mayor, todos los elementos de $A$ ser'an $m_1$. Se define adem'as $m_i$ como el mayor elemento de $A^i$. De 'esta forma
\begin{itemize}
\item $m_2 = \sum_{j=1}^k m_1\ m_1 = km_1^2$
\item $m_3 = \sum_{j=1}^k m_2\ m_2 = \sum_{j=1}^k k^2m_1^4 = k^3m_1^4$
\item \dots
\item $m_i = \sum_{j=1}^k m_{i-1}\ m_{i-1} = k^{2^i-1}m_1^{2^i}$
\end{itemize}

Dado que $n$ es potencia de 2, mientras que sea mayor a 1 el algoritmo ejecutar'a dos divisiones enteras (costando cada una O($\log_2(n)$)) y una multiplicaci'on matricial para elevar $pot2$ al cuadrado. Notar que en la iteraci'on $i$, $pot2$ ser'a $A^i$, por lo que todos los elementos de $pot2$ ser'an $m_i$. De 'esta forma, el costo de la multiplicaci'on ser'a del orden de O($k^3\log_2(m_i)$). 'Esto ser'a ejecutado para cada $1 \leq i \leq \log_2(n)$. Por lo tanto, la complejidad de 'estas multiplicaciones ser'a
$$\sum_{i=1}^{\log_2(n)} k^3\log_2(m_i) = k^3\sum_{i=1}^{\log_2(n)} \log_2(k^{2^i-1}m_1^{2^i}) = $$
$$k^3\left(\sum_{i=1}^{\log_2(n)} (2^i-1)\log_2(k) + \sum_{i=1}^{\log_2(n)}2^i\log_2(m_1)\right) = $$
$$k^3\log_2(k)\left(\sum_{i=1}^{\log_2(n)} 2^i\ \ -\ \ \log_2(n) \right)   +   k^3\log_2(m_1)\sum_{i=1}^{\log_2(n)}2^i = $$
$$k^3\log_2(k)\left(\sum_{i=0}^{\log_2(n)} 2^i\ \ -\ \ (\log_2(n)+1) \right)   +   k^3\log_2(m_1)\left(\sum_{i=0}^{\log_2(n)}2^i\ \ -\ \ 1\right) = $$
$$k^3\log_2(k)\left(\frac{2^{\log_2(n)+1} - 1}{2-1} - (\log_2(n)+1)\right) +  k^3\log_2(m_1)\left(\frac{2^{\log_2(n)+1}-1}{2-1} - 1\right) = $$
$$k^3\log_2(k)\left(2^{\log_2(n)+1} - 2 - \log_2(n) \right)   +  k^3 \log_2(m_1)\left(2^{\log_2(n)+1}-2\right) = $$
$$k^3\log_2(k)2^{\log_2(n)+1} - 2k^3\log_2(k) - k^3\log_2(k)\log_2(n) + k^3\log_2(m_1)2^{\log_2(n)+1} - 2k^3\log_2(m_1) \leq $$
$$k^3\log_2(k)2^{\log_2(n)+1} + k^3\log_2(m_1)2^{\log_2(n)+1} = $$
$$k^32^{\log_2(n)+1}(\log_2(k) + \log_2(m_1)) \in $$
$$O\left(k^32^{\log_2(n)}(\log_2(k) + \log_2(m_1))\right)$$

Si se agrega la complejidad de las dem'as operaciones, las $\log_2(n)$ iteraciones costar'an
$$O\left(k^32^{\log_2(n)}(\log_2(k) + \log_2(m_1))\right) + O\left(\log_2(n)(2\log_2(n))\right) = $$
$$O\left(k^32^{\log_2(n)}(\log_2(k) + \log_2(m_1)) + \log_2^2(n)\right) = O\left(k^32^{\log_2(n)}(\log_2(k) + \log_2(m_1))\right)$$
(no altera el resultado ya que $2^{\log_2(n)} \geq \log_2(n)$).

Luego har'a una 'ultima iteraci'on. Aqu'i suceder'a que $n=1$, se agregar'a una multiplicaci'on adicional entre $ret$ (quien todav'ia ser'a $I$) y $pot2$ (quien para 'ese entonces ser'a $A^n$). Dado que $A$ es una matriz de naturales, cada vez que se la eleva a alguna potencia natural todos sus valores mayores o iguales que los anteriores. Es por 'esto que se verifica que el m'aximo elemento en $pot2$ ser'a siempre mayor o igual que el m'aximo en $ret$ (en efecto, s'olo ser'an iguales si $A = I$). Por lo tanto, considero que todos los elementos de $ret$ son $m_n$, y la complejidad de 'esta multiplicaci'on ser'a 
$$O(k^3\log_2(m_n)) = O(k^3\log_2(k^{2^n-1}m_1^{2^n})) = O\left(k^3\left((2^n-1)\log_2(k)+2^n\log_2(m_1)\right)\right) =$$
$$O\left(k^32^n\left(\log_2(k)+\log_2(m_1)\right)\right)$$

Notar que la complejidad de 'esta sola multiplicaci'on es de mayor orden que todas las anteriores juntas. 'Esto se debe a que a medida que se eleva al cuadrado una matriz de naturales, sus valores aumentan r'apidamente. Por lo que todos los valores de $A^n$ ocupan un espacio mayor en orden que aquel ocupado por todos sus antecesores. As'i es como las operaciones son mucho m'as costosas. 'Esto podr'ia tenerse en mente a la hora de realizar implementaciones del algoritmo para evitar 'estas multiplicaciones ya que son innecesarias cuando $n$ es potencia de 2 (ya que es m'as eficiente copiar el valor de $pot2$ a $ret$ en lugar de hacer $I * A^n$)\footnote{El grupo decidi'o no implementarlas con el fin de mantener el algoritmo simple.}.

Juntando 'esta 'ultima complejidad con la de las dem'as iteraciones se obtiene que la complejidad logar'itmica de Potenciar es
$$O\left(k^32^{\log_2(n)}(\log_2(k) + \log_2(m_1))\right) + O\left(k^32^n\left(\log_2(k)+\log_2(m_1)\right)\right) = $$
$$O\left(k^32^n\left(\log_2(k)+\log_2(m_1)\right)\right)$$
(ya que $2^n \geq 2^{\log_2(n)}$)

Se demostr'o en el an'alisis uniforme que el orden del tama'no de la entrada es $t = k^2\log_2(m) + \log_2(n)$ (notar que $m_1 = m$).

Sabiendo que
$$2^t = 2^{k^2\log_2(m)+\log_2(n)} = \left(2^{\log_2(m)}\right)^{k^2} 2^{\log_2(n)} = m^{k^2}n$$
y que
$$2^{2^t} = 2^{m^{k^2}n} = \left(2^{n}\right)^{m^{k^2}}$$
se puede ver facilmente que
\begin{itemize}
\item $k^3 \leq 2^{2^t}$ ya que en el segundo, $k$ aparece como exponente y encima al cuadrado.
\item $2^n \leq 2^{2^t}$ ya que en el segundo aparece $2^n$ pero con a'un m'as exponentes.
\item $\log_2(k) \leq 2^{2^t}$ trivial a partir del primer 'item por ser $\log_2(m) \leq k^3$.
\end{itemize}

Como la 'unica variable que se encuentra en el exponente en $k^32^n\log_2(k)$ es $n$, todo 'ese producto nunca superar'a a $2^{2^t}$. Sucede lo mismo con $k^32^n\log_2(m)$, ya que aqu'i aparece $\log_2(m)$ pero 'este ser'a menor porque $m$ aparece en un exponente en $2^{2^t}$. Luego
$$k^32^n\log_2(k) + k^32^n\log_2(m) \leq 2^{2^t} + 2^{2^t} = 2^{2^t+1} \in O(2^{2^t})$$

Por lo que la complejidad logar'itmica se puede expresar como
$$O(2^{2^t})$$
con lo cual, 'esta es \negrita{exponencial} en funci'on del tama'no de la entrada.


\subsubsection{Fuerza Bruta}
El cambio con respecto a la fuerza bruta recae en Potenciar exclusivamente. Aqu'i, para calcular $A^n$ se ir'a multiplicando el resultado actual con $A$ hasta formar $A^n$. M'as exactamente, se realizar'an $n-1$ multiplicaciones. Por lo tanto, la complejidad ser'a O($(n-1)k^3$) = O($nk^3$).



%%%%%%%%%%%%%%%%%%%%%%%%%%%%%%%%%%%%%%%%%%%%%%%%%%%%%%%%%%%%%%%%%%%%%%
% Resultados
%%%%%%%%%%%%%%%%%%%%%%%%%%%%%%%%%%%%%%%%%%%%%%%%%%%%%%%%%%%%%%%%%%%%%%
\clearpage
\subsection{Resultados}
Los gr'aficos a continuaci'on cuentan la cantidad de operaciones de los algoritmos en funci'on de las variables de entrada. Se decidi'o as'i en lugar de hacerlos en funci'on del tama'no de la entrada para que sean sencillos de analizar.

\imagen{img/Tp1Ej3(multiplicar).png}{14}{Cantidad de operaciones b'asicas de Multiplicar en funci'on de las dimensiones de las matrices}
\imagen{img/Tp1Ej3(cant_mul).png}{14}{Cantidad de multiplicaciones realizadas en Potenciar en funci'on del exponente}
\imagen{img/Tp1Ej3(potenciar).png}{14}{Cantidad de operaciones b'asicas de Potenciar en funci'on del exponente y las dimensiones de las matrices}
\imagen{img/Tp1Ej3(potenciar_fb).png}{14}{Cantidad de operaciones b'asicas de Potenciar mediante el algoritmo 'optimo y el de fuerza bruta en funci'on del exponente y las dimensiones de las matrices}
\clearpage


%%%%%%%%%%%%%%%%%%%%%%%%%%%%%%%%%%%%%%%%%%%%%%%%%%%%%%%%%%%%%%%%%%%%%%
% Discusion
%%%%%%%%%%%%%%%%%%%%%%%%%%%%%%%%%%%%%%%%%%%%%%%%%%%%%%%%%%%%%%%%%%%%%%
\subsection{Discusi'on}
El primer gr'afico muestra que la complejidad te'orica y pr'actica son pr'acticamente la misma. Era de esperar un comportamiento as'i ya que el algoritmo es muy simple. Adem'as, la curva pr'actica est'a bien definida por ser el algoritmo O($k^3$) en cualquier caso, no simplemente el "peor" (porque, en realidad, aqu'i no existe peor caso). Las constantes a la hora de sumar los valores son los responsables de ese $5k^2$ adicional en comparaci'on con el caso te'orico (siguen estando en el mismo orden por ser $k^2 < k^3$).

En el segundo se ve la cantidad de multiplicaciones matriciales que realiza Potenciar en funci'on de $n$. Se verifica que en la pr'actica 'esta cantidad es exactamente $2\log_2(n) - 1$ en el peor caso. Es decir, es exactamente como se calcul'o te'oricamente. Notar que siempre el valor siguiente a un peor caso es de los mejores casos (es decir, cuando el exponente es potencia de 2) y que la cantidad de multiplicaciones en 'estos casos es $\log_2(n)+1$ como tambi'en se calcul'o te'oricamente.

El tercer gr'afico se apoya en los dos anteriores. 'Es decir, como las complejidades te'oricas anteriores fueron v'alidas, es de esperar que 'esta tambi'en lo sea ya que, como se ve en el an'alisis de la complejidad, se basa en 'estas dos. Y efectivamente, el orden de la complejidad te'orica es buena cota de la complejidad pr'actica.

Por 'ultimo, el cuarto gr'afico es el an'alisis contra el algoritmo de fuerza bruta. Como era de esperar, al necesitar m'as multiplicaciones el segundo que el primero la cantidad total de operaciones b'asicas realizadas es notablemente mayor a medida que $n$ y $k$ aumentan.
\clearpage


%%%%%%%%%%%%%%%%%%%%%%%%%%%%%%%%%%%%%%%%%%%%%%%%%%%%%%%%%%%%%%%%%%%%%%
% Conclusiones
%%%%%%%%%%%%%%%%%%%%%%%%%%%%%%%%%%%%%%%%%%%%%%%%%%%%%%%%%%%%%%%%%%%%%%
%\subsection{Conclusiones}
%????????????

\newpage
\section{Ejercicio 4}

%%%%%%%%%%%%%%%%%%%%%%%%%%%%%%%%%%%%%%%%%%%%%%%%%%%%%%%%%%%%%%%%%%%%%%
% Introduccion
%%%%%%%%%%%%%%%%%%%%%%%%%%%%%%%%%%%%%%%%%%%%%%%%%%%%%%%%%%%%%%%%%%%%%%
\subsection{Introducci'on}
Desde el principio, para resolver este problema se pens'o en armar la factorizaci'on probando todos los naturales (comenzando por el 2, ya que el 1 no es primo ni compuesto). Si 'este era primo, se verificaba si dividia al n'umero de entrada y se agregaban tantas repeticiones como fueran necesarias. Y ello fue lo que terminamos haciendo.

Para verificar si un n'umero era primo o no se utiliz'o el corolario de la criba de Erat'ostenes, es decir, probar si alg'un otro primo menor a su ra'iz lo divide. De no ser as'i, es porque efectivamente lo era; caso contrario, no. Pero 'este algoritmo era extremadamente ineficiente ya que deb'ia efectuar llamadas recursivas para cada natural menor a la ra'iz de la entrada para verificar si 'este era o no primo (y luego verificar si lo divid'ia). Por eso se opt'o por reutilizar mucha de 'esta informaci'on. 'Esto es, evitar preguntar si un n'umero es primo mas de una vez. Para ello se agreg'o una lista ordenada como entrada en donde dever'an estar todos los primos menores al valor que se est'a pasando. La ganancia fue del orden exponencial.


%%%%%%%%%%%%%%%%%%%%%%%%%%%%%%%%%%%%%%%%%%%%%%%%%%%%%%%%%%%%%%%%%%%%%%
% Detalles de implementacion
%%%%%%%%%%%%%%%%%%%%%%%%%%%%%%%%%%%%%%%%%%%%%%%%%%%%%%%%%%%%%%%%%%%%%%
\subsection{Detalles de implementaci'on}
En cierta manera, los algoritmos a implementar fueron relativamente sencillos. La mayor desici'on a tomar fue con respecto al tipo de lista utilizado. 

La lista $ret$ en Factorizaci'on tiene como prop'osito mantener valores en el orden que fueron agregados. Su acceso aleatorio no es relevante ya que en ning'un momento se toman valores ya guardados. En cambio, es cr'itico que el insertado sea en tiempo constante ya que la complejidad final ser'a afectada por la cantidad de operaciones que conlleva el insertado de cada factor. Es por 'esto que se eligi'o la lista doblemente enlazada, la cual tiene un acceso lineal pero un insertado constante.

La otra lista utilizada fue la de $primos$ en Factorizaci'on. El acceso aleatorio en 'este caso tampoco es importante porque los 'unicos accesos realizados son lineales (en EsPrimo), con lo que acceder al siguiente elemento es siempre O(1). Pero el insertado s'i que importa ya que si el valor a verificar efectivamente es primo, entonces hay que agregarlo. Por eso utilizamos tambi'en una lista doblemente enlazada.


%%%%%%%%%%%%%%%%%%%%%%%%%%%%%%%%%%%%%%%%%%%%%%%%%%%%%%%%%%%%%%%%%%%%%%
% Pseudocodigos
%%%%%%%%%%%%%%%%%%%%%%%%%%%%%%%%%%%%%%%%%%%%%%%%%%%%%%%%%%%%%%%%%%%%%%
\clearpage
\subsection{Pseudoc'odigos}

\algoritmo{Factorizacion($n$)}{Devuelve la factorizaci'on en primos de $n$}{O($n\sqrt{n}$)}
\begin{algorithmic}[1]
\REQUIRE $n \in \nat$ tal que $n \geq 2$
	\STATE $p \leftarrow 2$
	\WHILE{$n > 1$}
		\IF{EsPrimo($p,primos$)}
			\STATE $primos$.AgregarAtras($p$)
			\WHILE{$p\ |\ n$}
				\STATE $n \leftarrow n / p$
				\STATE $ret$.AgregarAtras($p$)
			\ENDWHILE
		\ENDIF
		\STATE $p \leftarrow p + 1$
	\ENDWHILE
	\RETURN $ret$
\end{algorithmic}

\vspace{2em}

\algoritmo{EsPrimo($n$)}{Decide si $n$ es o no primo}{O($\sqrt{n}$)}
\begin{algorithmic}[1]
\REQUIRE $primos$ tiene todos los primos enteros menores a $n$ ordenados crecientemente
	\STATE $r \leftarrow \lfloor \sqrt{n} \rfloor$;
		\FORALL{$p \leq r$ {\bf in} $primos$}
			\IF{$p\ |\ n$} \RETURN \FALSE \ENDIF
		\ENDFOR
	\RETURN \TRUE
\end{algorithmic}

\clearpage
%%%%%%%%%%%%%%%%%%%%%%%%%%%%%%%%%%%%%%%%%%%%%%%%%%%%%%%%%%%%%%%%%%%%%%
% Respuestas
%%%%%%%%%%%%%%%%%%%%%%%%%%%%%%%%%%%%%%%%%%%%%%%%%%%%%%%%%%%%%%%%%%%%%%
%\subsection{Respuestas}


%%%%%%%%%%%%%%%%%%%%%%%%%%%%%%%%%%%%%%%%%%%%%%%%%%%%%%%%%%%%%%%%%%%%%%
% Analisis de complejidad
%%%%%%%%%%%%%%%%%%%%%%%%%%%%%%%%%%%%%%%%%%%%%%%%%%%%%%%%%%%%%%%%%%%%%%
\subsection{An'alisis de complejidad}
\subsubsection{EsPrimo}

\subsubsubsection{Modelo Uniforme}

La idea de este algoritmo se basa en la Criba de Erat'ostenes. Es decir, afirma que si para cierto n'umero $n$ no existe un primo $p$ menor a $\sqrt{n}$ tal que $p\ |\ n$, entonces $n$ es primo. 

F'acilmente se ve que el peor caso es cuando $n$ efectivamente es primo, ya que fue necesario calcular todos los restos por divisi'on entera con los primos menores a $\sqrt{n}$. La mayor cantidad posible de valores le'idos es $\lfloor\sqrt{n}\rfloor$, y viene de suponer que todos los naturales menores a ese valor son primos.

Luego, la complejidad uniforme es O($\sqrt{n}$).

El tama'no de la entrada ser'a el de $n$ mas el de $primos$. El primero ser'a $\log_2(n)$, mientras que el segundo se puede acotar considerando que todos los valores de 2 a $n-1$ son primos. Como el logaritmo es creciente, el mayor de 'estos ser'a $n-1$. Asique, considerando que todos los valores son iguales a 'este, el tama'no de la lista quedar'ia como
$$\sum_{i=2}^{n-1} \log_2(n-1) = (n-2)\log_2(n-1) \in O(n\log_2(n))$$

Agreg'ando el tama'no de $n$ quedar'ia
$$t = O(n\log_2(n) + \log_2(n)) = O(n\log_2(n))$$

Por lo que la complejidad uniforme se podr'ia expresar como O($\sqrt{t}$), con lo que terminar'ia siendo \negrita{sublineal} en funci'on de la entrada.

\vspace{2em}

\subsubsubsection{Modelo Logar'itmico}

Considero $n$ primo por ser 'este el peor caso. Calcular $\sqrt{n}$ es considerada que cuesta O($\log_2(n)$). Luego se deber'a iterar para cada primo menor o igual que $\sqrt{n}$, por lo que considero que ciclar'a $\sqrt{n}$ veces. Para cada iteraci'on, se deber'an realizar una comparaci'on y una divisi'on entera (en ralidad se calcula el resto, pero es pr'acticamente lo mismo), costando cada una de ellas O($\log_2(n)$). 

Luego, la complejidad logar'itmica es O($\sqrt{n}\log_2(n)$).

El tama'no de entrada es $t = O(n\log_2(n))$ y f'acilmente se verifica que es mayor a la complejidad logar'itmica. Asique 'esta ser'a O($t$), con lo que es \negrita{lineal} en funci'on del tama'no de la entrada.


\subsubsection{Factorizacion}

\subsubsubsection{Modelo Uniforme}

'Este algoritmo devuelve una lista de naturales, ordenados de menor a mayor, representando cada uno a un producto de la factorizaci'on de $n$. 

Para calcularla, comienza por el menor primo (el 2) y verifica si lo es. De serlo, lo agrega a una lista (utilizada luego para probar si los siguientes valores son primos) y comienza a agregarlo a la lista de factores tantas veces como divida a $n$. Luego, incrementa el natural a probar y repite el procedimiento hasta que se hallan obtenido todos los divisores primos.

El peor de los casos es cuando $n$ es primo, ya que se deber'a llamar a EsPrimo $n$ veces (el 'unico divisor que se encontrar'a es $n$). De ser un n'umero compuesto, se llamar'a a lo sumo $n/t$ veces a 'esta funci'on, siendo $t$ el menor primo en la factorizaci'on. Adem'as, el bucle interno que verifica cuantas veces divide un primo a $n$ nunca se ejecutar'a en total (o sea, sumando para todos los primos divisibles) m'as de $\log_p{n}$ veces, siendo $p$ el m'aximo primo en la factorizaci'on.

Por lo tanto, como EsPrimo se ejecuta a lo sumo $n$ veces, y en cada llamado cuesta O($\sqrt{n}$) operaciones, entonces la complejidad de Factorizacion es O($n\sqrt{n}$).

El tama'no de la entrada ser'a simplemente $t = \log_2(n)$. Notar que $2^{\log_2(n)} = 2^t = n$, por lo que la complejidad uniforme de Factorizacion se puede expresar como 
$$O(2^t\sqrt{2^t}) = O(2^{3t/2})$$
con lo que ser'ia \negrita{exponencial} en funci'on del tama'no de la entrada.

\vspace{2em}

\subsubsubsection{Modelo Logar'itmico}

Se supone nuevamente que $n$ es primo por ser el peor caso. Entonces, se deber'a iterar $n-1$ veces realizando en cada ciclo:
\begin{itemize}
\item Una comparaci'on contra 1 $\rightarrow$ O($\log_2(n)$).
\item Una llamada a EsPrimo  $\rightarrow$ O($\sqrt{n}\log_2(n)$).
\item Una suma a $p$, el cual nunca es mayor a $n$  $\rightarrow$ O($\log_2(n)$).
\item Cuando $p = n$, en el bucle interno se realizan dos divisiones enteras y dos AgregarAtras  $\rightarrow$ O($4\log_2(n)$) = O($\log_2(n)$).
\end{itemize}

Por lo que cada iteraci'on costar'a la suma de todos los 'ordenes: O($\sqrt{n}\log_2(n)$).

Luego, la complejidad de Factorizacion es O($n\sqrt{n}\log_2(n)$).

De forma equivalente al modelo uniforme, siendo $t = \log_2(n)$ el tama'no de la entrada, la complejidad logar'itmica se puede expresar como 
$$O(2^t\sqrt{2^t}t) = O(2^{3t/2}t)$$
con lo que 'esta ser'ia \negrita{exponencial} en funci'on del tama'no de la entrada.

\subsubsection{Fuerza Bruta}
Las diferencias entre el algoritmo de fuerza bruta y el implementado son principalmente dos en EsPrimo:
\begin{itemize}
\item El algoritmo verificar'a si alg'un primo menor a $n$ lo divide, en lugar de verificar hasta $\sqrt{n}$.
\item No posee una lista de primos menores a $n$, por lo que deber'a verificar para cada natural $p$ desde 2 hasta $n$ si es o no primo, donde a su vez deber'a verificar tambi'en si alg'un primo de 2 a $p-1$ lo divide, etc.
\end{itemize}
As'i es como la cantidad de llamadas recursivas ser'a expresada en funci'on de la cantidad de llamadas de los valores anteriores (suponiendo que todos los n'umeros de 2 a $n$ son primos). i.e.
\begin{itemize}
\item $T(2) = 0$
\item $T(3) = T(2) + 1 = 0 + 1 = 1$
\item $T(4) = T(2) + T(3) + 2 = 0 + 1 + 2 = 3$
\item $T(5) = T(2) + T(3) + T(4) + 3 = 0 + 1 + 3 + 3 = 7$
\item $T(6) = T(2) + T(3) + T(4) + T(5) + 4 = 0 + 1 + 3 + 7 + 4 = 15$
\item $T(7) = T(2) + T(3) + T(4) + T(5) + T(6) + 5 = 0 + 1 + 3 + 7 + 15 + 5 = 31$
\item $\dots$
\item $T(n) = \sum_{i=2}^{n-1}{T(i)} + (n-2)$
\end{itemize}

El t'ermino general de la recursi'on es
$$T(n) = \sum_{i=2}^{n-1}{2^{i-2}} = \sum_{j=0}^{n-3}{2^j} = \frac{2^{n-2}-1}{2-1} = 2^{n-2} - 1$$

Demostraci'on por inducci'on sobre $n$:

\underline{Caso base ($n$ = 2)}:
$$T(2) = 2^{2-2} - 1 = 0$$

\underline{Caso inductivo (se cumple para $T(n-1) \Longrightarrow$ se cumple para $T(n)$)}:
$$T(n) = \sum_{i=2}^{n-1}{T(i)} + (n-2) = \sum_{i=2}^{n-2}{T(i)} + T(n-1) + (n-3+1) =$$
$$\sum_{i=2}^{n-2}{T(i)} + (n-3) + T(n-1) + 1 = T(n-1) + T(n-1) + 1 =  (HI)$$
$$2(2^{n-3} - 1) + 1 = 2^{n-2} - 1 $$
\begin{flushright}
$\Box$   
\end{flushright}

Por lo tanto la complejidad de EsPrimo por fuerza bruta es (en el peor caso)
$$O(2^{n-2}-1) = O(2^n)$$

con lo que de Factorizar por fuerza bruta es del orden deO($n2^n$).



%%%%%%%%%%%%%%%%%%%%%%%%%%%%%%%%%%%%%%%%%%%%%%%%%%%%%%%%%%%%%%%%%%%%%%
% Resultados
%%%%%%%%%%%%%%%%%%%%%%%%%%%%%%%%%%%%%%%%%%%%%%%%%%%%%%%%%%%%%%%%%%%%%%
\clearpage
\subsection{Resultados}
\imagen{img/Tp1Ej4(es_primo).png}{14}{Cantidad de operaciones b'asicas de EsPrimo en funci'on del n'umero de entrada}
\imagen{img/Tp1Ej4(cant_ep).png}{14}{Cantidad de llamadas a la funci'on EsPrimo desde Factorizacion en funci'on del n'umero a factorizar}
\imagen{img/Tp1Ej4(factorizacion).png}{14}{Cantidad de operaciones b'asicas de Factorizacion en funci'on del n'umero a factorizar}
\imagen{img/Tp1Ej4(factorizacion_fb).png}{14}{Cantidad de operaciones b'asicas de Factorizacion mediante el algoritmo 'optimo y el de fuerza bruta en funci'on del n'umero a factorizar}
\clearpage


%%%%%%%%%%%%%%%%%%%%%%%%%%%%%%%%%%%%%%%%%%%%%%%%%%%%%%%%%%%%%%%%%%%%%%
% Discusion
%%%%%%%%%%%%%%%%%%%%%%%%%%%%%%%%%%%%%%%%%%%%%%%%%%%%%%%%%%%%%%%%%%%%%%
\subsection{Discusi'on}
En el primer gr'afico se puede ver que la cota de O($\sqrt{n}$) es v'alida, ya que para cierto valor inicial 'esta funci'on siempre es mayor que la cantidad de operaciones reales. Pero puede parecer exagerada. 'Esto es perfectamente esperable debido a que en el c'alculo de la complejidad se consider'o como peor caso de EsPrimo aquel donde {\bf todos} los valores menores a $\sqrt{n}$ eran primos, y por supuesto 'esto es imposible (excepto, obviamente, con valores peque'nos). De existir una mejor forma para acotar la cantidad de primos menores a $\sqrt{n}$, entonces la complejidad te'orica podr'ia ser mejor ajustada.

En el segundo se notan patrones peculiares en la distribuci'on de los valores: ciertas rectas de diferentes pendientes. Despu'es de analizar algunos puntos de cada una se descubri'o que:
\begin{itemize}
\item La mayor son la cantidad de llamados que se hicieron a EsPrimo para valores que efectivamente eran primos. Es m'as, los tiempos de {\bf todos} los primos calleron aqu'i. Dado que la distribuci'on es id'entica a la funcion $n$, nos demuestra que el an'alisis te'orico fue correcto: si el valor $n$ a factorizar es primo entonces la cantidad de valores consultados por EsPrimo ser'a del orden de $n$.
\item La recta que le sigue a la anterior tiene la mitad de pendiente que la anterior, es decir, sigue la funci'on $n/2$. Los valores para los cuales se hacen esta cantidad de llamadas a EsPrimo es cuando en su factorizaci'on aparecen s'olo dos productos, y uno de ellos es el 2. 'Esto tiene mucho sentido debido a que, al encontrar el primer primo f'acilmente (es el primero) el algoritmo deber'a correr hasta encontrar al primo restante. Al no ser 'este compuesto, la cantidad de llamados a EsPrimo que le restan por hacer ser'a del orden de 'este valor. Por lo tanto, la cantidad de llamadas ser'a del orden de $n/2$, lo cual justifica la distribuci'on adoptada.
\item La siguiente recta tiene pendiente $n/3$, y similarmente al caso anterior, aqu'i recaen aquellos valores que en su factorizaci'on est'a el 3 y otro primo.
\item La siguiente es $n/4$ y los valores tienen una factorizaci'on con el 2 dos veces y luego otro primo.
\item etc\dots
\end{itemize}

El tercer gr'afico se basa en los dos anteriores. El patr'on de rectas se sigue notando y se debe a la cantidad de llamados a EsPrimo. La cota te'orica es v'alida pero, debido a que la cota en EsPrimo era elevada y la primera se basa en la segunda, aqu'i tambi'en termina si'endolo.

Por 'ultimo, en el cuarto gr'afico se puede ver que nuesto algoritmo es mucho m'as eficiente en comparaci'on con el de fuerza bruta. Y era de esperar ya que nuestro algoritmo es polinomial y el de fuerza bruta, exponencial.

\clearpage

%%%%%%%%%%%%%%%%%%%%%%%%%%%%%%%%%%%%%%%%%%%%%%%%%%%%%%%%%%%%%%%%%%%%%%
% Conclusiones
%%%%%%%%%%%%%%%%%%%%%%%%%%%%%%%%%%%%%%%%%%%%%%%%%%%%%%%%%%%%%%%%%%%%%%
%\subsection{Conclusiones}
%?????????????????????????????????



%%%%%%%%%%%%%%%%%%%%%%%%%%%%%%%%%%%%%%%%%%%%%%%%%%%%%%%%%%%%%%%%%%%%%%
% Enunciado
%%%%%%%%%%%%%%%%%%%%%%%%%%%%%%%%%%%%%%%%%%%%%%%%%%%%%%%%%%%%%%%%%%%%%%
%\newpage
%\section*{Enunciado}
%----------------------------------------------------------------------------------%
%??????????????????
%----------------------------------------------------------------------------------%
%\newpage


%%%%%%%%%%%%%%%%%%%%%%%%%%%%%%%%%%%%%%%%%%%%%%%%%%%%%%%%%%%%%%%%%%%%%%
% Firmas
%%%%%%%%%%%%%%%%%%%%%%%%%%%%%%%%%%%%%%%%%%%%%%%%%%%%%%%%%%%%%%%%%%%%%%
% ????????

%%%%%%%%%%%%%%%%%%%%%%%%%%%%%%%%%%%%%%%%%%%%%%%%%%%%%%%%%%%%%%%%%%%%%%
% Referencias
%%%%%%%%%%%%%%%%%%%%%%%%%%%%%%%%%%%%%%%%%%%%%%%%%%%%%%%%%%%%%%%%%%%%%%
\section{Referencias}
\begin{thebibliography}{99}

	\bibitem{strassen} Algoritmo de Strassen para la multiplicaci'on de matrices cuadradas:\\  {\it http://en.wikipedia.org/wiki/Strassen\_algorithm}

	\bibitem{cw} Algoritmo de Coppersmith-�Winograd para la multiplicaci'on de matrices cuadradas:\\  {\it http://en.wikipedia.org/wiki/Coppersmith\%E2\%80\%93Winograd\_algorithm}

	\bibitem{binpow} Golub \& Van Loan, {\it Matrix Computations}; secci'on 11.2.5, p'agina 569.

\end{thebibliography}



\end{document}
%%%
% EOF
%%%%%%%%%%%%%%%%%%%%%
