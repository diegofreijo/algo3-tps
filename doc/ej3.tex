\newpage
\section{Ejercicio 3}

%%%%%%%%%%%%%%%%%%%%%%%%%%%%%%%%%%%%%%%%%%%%%%%%%%%%%%%%%%%%%%%%%%%%%%
% Introduccion
%%%%%%%%%%%%%%%%%%%%%%%%%%%%%%%%%%%%%%%%%%%%%%%%%%%%%%%%%%%%%%%%%%%%%%
\subsection{Introducci'on}
Al comenzar a pensar la soluci'on de 'este ejercicio, partimos de un algortimo por fuerza bruta e intentamos mejorarlo. Cre'imos que el punto clave en donde se podr'ian ahorrar operaciones era en la multiplicaci'on de matrices. Luego de varios intentos e investigaci'ones por internet, hallamos que exist'ian algoritmos de multiplicaci'on de matrices cuadradas m'as r'apidos que el {\it standart} de orden c'ubico sobre las dimensiones de la matriz. Por ejemplo, el algoritmo de Strassen\superindice{\cite{strassen}} del orden de O($k^{2.807}$) y el de Coppersmith-�Winograd\superindice{\cite{cw}} del orden de O($k^{2.376}$). Pero ninguno de ellos fue utilizado. La desici'on provino de notar que los algoritmos eran dif'iciles de implementar y solamente para obtener una \emph{peque'na} disminuci'on en la complejidad. Adem'as, los c'alculos de complejidad (especialmente en el modelo logar'itmico) se hubiesen complejizado demasiado. Evidentemente, el gran salto entre la fuerza bruta y un algoritmo m'as eficiente no estaba all'i.

Entonces se pens'o en atacar el problema m'as desde arriba. Cada multiplicaci'on entre matrices es una operaci'on muy costosa, por lo que nos pareci'o interesante lograr disminu'ir el n'umero de 'estas al realizar la potenciaci'on. La primer idea que surgi'o fue de suponer al exponente como potencia de 2, por ejemplo 4. Por fuerza bruta, el algoritmo hubiese hecho
$$A = A*A*A*A$$
costando tres multiplicaciones (complejidad lineal). Pero 'esto se puede mejorar, ya que
$$A = (A^2)^2$$
con lo que la cantidad de multiplicaciones requeridas hubiesen sido una para multiplicar $A$ por s'i misma ($A^2$) y otra para multiplicar el resultado por s'i mismo ($(A^2)^2$), disminuyendo en una tercera parte la cantidad de multiplicaciones. En efecto, la complejidad resultante (contando solamente multiplicaciones) ser'ia logar'itmica. Pero hab'ia un serio problema de fondo: el exponente no ten'ia porque ser potencia de dos.

Con ello en mente, buscamos informaci'on sobre como aplicar la misma idea a cualquier exponente. Y nos encontramos con el algortimo de potenciaci'on binaria ({\it binary power})\superindice{\cite{binpow}}. 'Este utiliza la misma idea que ten'iamos pero salva los casos que no son potencia de dos tomando la representaci'on binaria del exponente. As'i es como funciona:

\begin{itemize}
	\item Suponer que quiero realizar $A^n$.
	\item Suponer que la representaci'on binaria de $n$ es $b_{k-1}...b_0$, donde cada $b_i$ es un d'igito binario.
	\item Claramente 
\begin{equation}
	\label{sumabin}
	n = \sum_{i=0}^{k-1} b_i  2^i
\end{equation}
o sea que para formar a $n$ debo sumar aquellas potencias de 2 en donde su d'igito binario de $n$ es 1.
	\item De 'esta forma es f'acil construirse el algoritmo. La matriz resultado comienza siendo la identidad (considerar que equivale a $A^0$). A su vez tengo otra matriz con exponentes de $A$ potencias de dos (comienza siendo $A^{2^0} = A$). Voy tomando los valores de $b_0$ en adelante y en donde halla un uno, multiplico la matriz resultado por la de las potencias (ya que la multiplicaci'on entre las matrices repercute como una suma entre los exponentes). Al final de cada iteraci'on elevo al cuadrado la matriz de las potencias de 2 para pasar a la siguiente. Repitiendo con cada $b_i$ es como voy transformando el exponente del resultado en $n$. Al terminar con $b_{k-1}$, 'este va a ser efectivamente $n$ (y la justificaci'on es la sumatoria expresada en (\ref{sumabin})).
\end{itemize}

'Este algoritmo es del orden de $2\log_2(n)$ multiplicaciones en el peor caso, con lo que nos quedamos bastante satisfechos y fue el utilizado.


%%%%%%%%%%%%%%%%%%%%%%%%%%%%%%%%%%%%%%%%%%%%%%%%%%%%%%%%%%%%%%%%%%%%%%
% Detalles de implementacion
%%%%%%%%%%%%%%%%%%%%%%%%%%%%%%%%%%%%%%%%%%%%%%%%%%%%%%%%%%%%%%%%%%%%%%
\subsection{Detalles de implementaci'on}
Para facilitar la implementaci'on lo primero que se hizo fue una clase Matriz, la cual representa una matriz cuadrada y posee solamente los m'etodos necesarios para resolver 'este ejercicio (principalmente, Multiplicar y Potenciar).

Al algoritmo Potenciar se le hicieron peque'nos cambios al implementarlo con el 'unico fin de hacerlo m'as eficiente:
\begin{itemize}
\item La representaci'on binaria de $n$ no se calcula antes de comenzar a multiplicar matrices, sino que se realizan las dos operaciones a la vez. Se utiliz'o el algoritmo {\it standart} para obtenerla.
\item Se agreg'o una guarda a la multiplicaci'on de $pot2$ para evitar que se realize una multiplicaci'on in'util cuando ya se alcanz'o el resultado. 'Esto suma O($\log_2(n)$) operaciones b'asicas a la complejidad final del algoritmo, pero evita sumar el O($k^3$) de la complejidad de una multiplicaci'on, por lo que en la pr'actica genera una notable optimizaci'on.
\end{itemize}



%%%%%%%%%%%%%%%%%%%%%%%%%%%%%%%%%%%%%%%%%%%%%%%%%%%%%%%%%%%%%%%%%%%%%%
% Pseudocodigos
%%%%%%%%%%%%%%%%%%%%%%%%%%%%%%%%%%%%%%%%%%%%%%%%%%%%%%%%%%%%%%%%%%%%%%
\clearpage
\subsection{Pseudoc'odigos}

\algoritmo{Potenciar($A,n$)}{Devuelve $A^n$}{O($log_2(n)k^3$)}
\begin{algorithmic}[1]
\REQUIRE $A \in \nat^{k \times k}$
\IF{$n = 0$}
	\RETURN $I \in \nat^{k \times k}$
\ENDIF
\STATE $ret \leftarrow I$
\STATE $pot2 \leftarrow A$
\WHILE{$n \ge 1$}
	\IF{$2\ |\ n$}
		\STATE $n \leftarrow n\ /\ 2$ 
	\ELSE
		\STATE $n \leftarrow (n-1)\ /\ 2$
		\STATE $ret \leftarrow$ Multiplicar($ret,pot2$)
	\ENDIF
	\STATE $pot2 \leftarrow$ Multiplicar($pot2,pot2$) 
\ENDWHILE
\RETURN $ret$
\end{algorithmic}

\vspace{2em}

\algoritmo{Multiplicar($A,B$)}{Multiplica la matriz $A$ por $B$}{O($k^3$)}
\begin{algorithmic}[1]
\REQUIRE $A,B \in \nat^{k \times k}$
	\STATE $ret \in \nat^{k \times k}$
	\FOR{$i = 1 \dots k$}
		\FOR{$j = 1 \dots k$}
			\STATE $ret_{ij} \leftarrow \sum_{t=1}^{k}{A_{it}B_{tj}}$
		\ENDFOR
	\ENDFOR
	\RETURN $ret$
\end{algorithmic}


\clearpage


%%%%%%%%%%%%%%%%%%%%%%%%%%%%%%%%%%%%%%%%%%%%%%%%%%%%%%%%%%%%%%%%%%%%%%
% Analisis de complejidad
%%%%%%%%%%%%%%%%%%%%%%%%%%%%%%%%%%%%%%%%%%%%%%%%%%%%%%%%%%%%%%%%%%%%%%
\subsection{An'alisis de complejidad}
\subsubsection{Multiplicar}

\subsubsubsection{Modelo Uniforme}

F'acilmente se puede ver en el algoritmo que para cada fila de $A$ (las cuales son $k$) y cada columna de $B$ (las cuales tambi'en son $k$) se debe recorrer cada elemento de ellas (los cuales, nuevamente, son $k$). Notar que siempre realiza lo mismo, sin importar de $k$ o los valores en la matriz, por lo que no hay mejor ni peor caso.
 
Luego, la complejidad uniforme es O($k^3$).

El tama'no de la entrada ser'a igual a la suma de los tama'nos de todos los elementos de cada matriz:
$$t = \sum_{i=1}^k\sum_{j=1}^k \log_2(A_{ij})+\log_2(B_{ij})$$

Pero como, para acotar, se considera que todos los elementos de cada una son el mayor de ambas (llamado $m$), el tama'no se puede expresar como
$$t = \sum_{i=1}^k\sum_{j=1}^k \log_2(m)+\log_2(m) = 2k^2\log_2(m) \in O(k^2\log_2(m))$$

Por lo tanto, la complejidad uniforme es O($\sqrt{t/2}\ t/2$) = O($t^{3/2}$), con lo que es \negrita{superlineal} en funci'on del tama'no de entrada.

\vspace{2em}

\subsubsubsection{Modelo Logar'itmico}

Para facilitar los c'alculos, se considerar'a que todos los valores de la matriz ser'an el m'as grande de todos y el que, por consiguiente, m'as espacio ocupe. 'Esto no afectar'a la validez del resultado ya que se estar'a calculando un orden, una cota superior. Supongo que $m$ es 'este valor. Luego, el espacio que ocupar'a su representaci'on binaria ser'a $\lceil \log_2(m+1) \rceil$. Como es del orden de O($\log_2(m)$), simplemente se considerar'a a 'este como el tama'no de $m$. De 'esta forma, realizar $m+m$ y $m*m$ se considerar'a que cuestan O($2\log_2(m)$) = O($\log_2(m)$) cada operaci'on.

En Multiplicar, para calcular cada valor del resultado se deben multiplicar $m$ con $m$ tantas veces como la dimension de la matriz. 'Estos productos deber'an ser sumados entre s'i. Primero $m^2+m^2$, luego $2m^2+m^2$, luego $3m^2+m^2$, etc. Es decir que cada suma costar'a 
$$ O(\log_2(cm^2)+\log_2(m^2)) = O(2\log_2(m) + \log_2(c) + 2\log_2(m)) = O(\log_2(m))$$

Por lo tanto, para calcular un valor de la matriz resultado, se deber'an calcular $k$ productos cada uno con un costo de O($\log_2(m)$) y $k-1$ sumas cada una con el mismo costo. 'Esto se deber'a realizar para cada elemento de la matriz resultado, la cual contiene $k^2$ de ellos. Luego, la complejidad logar'itmica de Multiplicar estar'a dada por
$$O((k\log_2(m) + (k-1)\log_2(m))k^2) = O(k^3\log_2(m))$$

Se verifica f'acilemnte que
$$t^2 \geq k^3\log_2(m) \Longleftrightarrow k^4\log^2_2(m) \geq k^3\log_2(m) \Longleftrightarrow k \geq 1$$

Por lo tanto, la complejidad logar'itmica se puede expresar como
$$O(t^2)$$
con lo que ser'a \negrita{cuadr'atica} en funci'on del tama'no de la entrada.


\subsubsection{Potenciar}

\subsubsubsection{Modelo Uniforme}

Dado que una multiplicaci'on entre matrices es muy costosa, y que es la m'as costosa en 'este algoritmo\footnote{Notar que los costos de copiar $A$ en $pot2$ e $I$ en $ret$ son del orden de O($k^2$) cada una, pero como s'olo se ejecutan una vez, y la complejidad de Multiplicar es mayor por ser O($k^3$), se desprecian.},  el peor de los casos para Potenciar ser'a cuando deba realizar la mayor cantidad de 'estas. Como $pot2$ siempre ser'a elevada al cuadrado, en cada iteraci'on se realizar'a por lo menos una multiplicaci'on. Pero la otra, correspondiente a $ret * pot2$, s'olo cuando el $n$ actual sea impar. 'Esto, como mejor se explic'o en la Introducci'on del presente, sucede si se encuentra un 1 en la representaci'on binaria de $n$. 

Por lo tanto, el peor $n$ de entrada ser'a aquel con todos unos en su representaci'on binaria. Y 'esto es equivalente a decir que
$$n = 2^r-1$$
para alg'un $r\in\nat$.

Ya que con un $n$ as'i se realizar'an 2 multiplicaciones por cada d'igito binario suyo (excepto en la 'ultima iteraci'on que realiza s'olo una), y que la cantidad de 'estos es $\log_2(n)$, la cantidad de multiplicaciones ser'an O($2*\log_2(n) - 1$). Pero, como cada multiplicaci'on es O($k^3$), la complejidad uniforme de Potenciar ser'a
$$O(2\log_2(n)k^3) = O(k^3\log_2(n))$$

Notar que el mejor caso ser'a aquel donde $n$ tenga la mayor cantidad de ceros posible en su representaci'on binaria. 'Es decir, $n = 100\dots 00$. All'i, s'olo realizar'ia $\log_2(n)$ multiplicaciones para $pot2$ y luego una m'as para hacer $ret*pot2$ (que en realidad ser'ia $I*A^n$). Por lo tanto, el mejor caso ser'a con $n$ potencia de dos.

El tama'no de la entrada ser'a igual a la suma del tama'no de todos los elementos de $A$ mas el de $n$. Como se considera que todos los elementos de $A$ son el m'aximo de ellos (llamado $m$), el tama'no de la entrada ser'a
$$t = \log_2(n) + \sum_{i=1}^k\sum_{j=1}^k \log_2(m) = k^2\log_2(m) + \log_2(n)$$

Tratando de acotar la complejidad, se puede ver primero que
$$t^2 = (k^2\log_2(m) + \log_2(n))^2 = $$
$$k^4\log^2_2(m) + 2k^2\log_2(m)\log_2(n) + \log^2_2(n) \geq k^2\log_2(n)$$
y que
$$\sqrt{t} = \sqrt{k^2\log_2(m) + \log_2(n)} \geq \sqrt{k^2} = k$$

Por lo tanto, la complejidad puede ser expresada como
$$O(t^2\sqrt{t}) = O(t^{5/2})$$
con lo que 'esta ser'ia \negrita{supercuadr'atica}\footnote{Nombre generado por inducci'on en \negrita{superlineal}. Tranquilamente se podr'ia afirmar que es \negrita{c'ubica}.} en funci'on del tama'no de la entrada.

\vspace{2em}

\subsubsubsection{Modelo Logar'itmico}
Dado el enunciado, se considerar'a que $n = 2^r$ para alg'un $r\in\nat$, adem'as de las suposiciones hechas en el c'alculo de Multiplicar.

Sea $m_1$ el mayor elemento de $A$. Como se considera que todos los elementos de la matriz son el mayor, todos los elementos de $A$ ser'an $m_1$. Se define adem'as $m_i$ como el mayor elemento de $A^i$. De 'esta forma
\begin{itemize}
\item $m_2 = \sum_{j=1}^k m_1\ m_1 = km_1^2$
\item $m_3 = \sum_{j=1}^k m_2\ m_2 = \sum_{j=1}^k k^2m_1^4 = k^3m_1^4$
\item \dots
\item $m_i = \sum_{j=1}^k m_{i-1}\ m_{i-1} = k^{2^i-1}m_1^{2^i}$
\end{itemize}

Dado que $n$ es potencia de 2, mientras que sea mayor a 1 el algoritmo ejecutar'a dos divisiones enteras (costando cada una O($\log_2(n)$)) y una multiplicaci'on matricial para elevar $pot2$ al cuadrado. Notar que en la iteraci'on $i$, $pot2$ ser'a $A^i$, por lo que todos los elementos de $pot2$ ser'an $m_i$. De 'esta forma, el costo de la multiplicaci'on ser'a del orden de O($k^3\log_2(m_i)$). 'Esto ser'a ejecutado para cada $1 \leq i \leq \log_2(n)$. Por lo tanto, la complejidad de 'estas multiplicaciones ser'a
$$\sum_{i=1}^{\log_2(n)} k^3\log_2(m_i) = k^3\sum_{i=1}^{\log_2(n)} \log_2(k^{2^i-1}m_1^{2^i}) = $$
$$k^3\left(\sum_{i=1}^{\log_2(n)} (2^i-1)\log_2(k) + \sum_{i=1}^{\log_2(n)}2^i\log_2(m_1)\right) = $$
$$k^3\log_2(k)\left(\sum_{i=1}^{\log_2(n)} 2^i\ \ -\ \ \log_2(n) \right)   +   k^3\log_2(m_1)\sum_{i=1}^{\log_2(n)}2^i = $$
$$k^3\log_2(k)\left(\sum_{i=0}^{\log_2(n)} 2^i\ \ -\ \ (\log_2(n)+1) \right)   +   k^3\log_2(m_1)\left(\sum_{i=0}^{\log_2(n)}2^i\ \ -\ \ 1\right) = $$
$$k^3\log_2(k)\left(\frac{2^{\log_2(n)+1} - 1}{2-1} - (\log_2(n)+1)\right) +  k^3\log_2(m_1)\left(\frac{2^{\log_2(n)+1}-1}{2-1} - 1\right) = $$
$$k^3\log_2(k)\left(2^{\log_2(n)+1} - 2 - \log_2(n) \right)   +  k^3 \log_2(m_1)\left(2^{\log_2(n)+1}-2\right) = $$
$$k^3\log_2(k)2^{\log_2(n)+1} - 2k^3\log_2(k) - k^3\log_2(k)\log_2(n) + k^3\log_2(m_1)2^{\log_2(n)+1} - 2k^3\log_2(m_1) \leq $$
$$k^3\log_2(k)2^{\log_2(n)+1} + k^3\log_2(m_1)2^{\log_2(n)+1} = $$
$$k^32^{\log_2(n)+1}(\log_2(k) + \log_2(m_1)) \in $$
$$O\left(k^32^{\log_2(n)}(\log_2(k) + \log_2(m_1))\right)$$

Si se agrega la complejidad de las dem'as operaciones, las $\log_2(n)$ iteraciones costar'an
$$O\left(k^32^{\log_2(n)}(\log_2(k) + \log_2(m_1))\right) + O\left(\log_2(n)(2\log_2(n))\right) = $$
$$O\left(k^32^{\log_2(n)}(\log_2(k) + \log_2(m_1)) + \log_2^2(n)\right) = O\left(k^32^{\log_2(n)}(\log_2(k) + \log_2(m_1))\right)$$
(no altera el resultado ya que $2^{\log_2(n)} \geq \log_2(n)$).

Luego har'a una 'ultima iteraci'on. Aqu'i suceder'a que $n=1$, se agregar'a una multiplicaci'on adicional entre $ret$ (quien todav'ia ser'a $I$) y $pot2$ (quien para 'ese entonces ser'a $A^n$). Dado que $A$ es una matriz de naturales, cada vez que se la eleva a alguna potencia natural todos sus valores mayores o iguales que los anteriores. Es por 'esto que se verifica que el m'aximo elemento en $pot2$ ser'a siempre mayor o igual que el m'aximo en $ret$ (en efecto, s'olo ser'an iguales si $A = I$). Por lo tanto, considero que todos los elementos de $ret$ son $m_n$, y la complejidad de 'esta multiplicaci'on ser'a 
$$O(k^3\log_2(m_n)) = O(k^3\log_2(k^{2^n-1}m_1^{2^n})) = O\left(k^3\left((2^n-1)\log_2(k)+2^n\log_2(m_1)\right)\right) =$$
$$O\left(k^32^n\left(\log_2(k)+\log_2(m_1)\right)\right)$$

Notar que la complejidad de 'esta sola multiplicaci'on es de mayor orden que todas las anteriores juntas. 'Esto se debe a que a medida que se eleva al cuadrado una matriz de naturales, sus valores aumentan r'apidamente. Por lo que todos los valores de $A^n$ ocupan un espacio mayor en orden que aquel ocupado por todos sus antecesores. As'i es como las operaciones son mucho m'as costosas. 'Esto podr'ia tenerse en mente a la hora de realizar implementaciones del algoritmo para evitar 'estas multiplicaciones ya que son innecesarias cuando $n$ es potencia de 2 (ya que es m'as eficiente copiar el valor de $pot2$ a $ret$ en lugar de hacer $I * A^n$)\footnote{El grupo decidi'o no implementarlas con el fin de mantener el algoritmo simple.}.

Juntando 'esta 'ultima complejidad con la de las dem'as iteraciones se obtiene que la complejidad logar'itmica de Potenciar es
$$O\left(k^32^{\log_2(n)}(\log_2(k) + \log_2(m_1))\right) + O\left(k^32^n\left(\log_2(k)+\log_2(m_1)\right)\right) = $$
$$O\left(k^32^n\left(\log_2(k)+\log_2(m_1)\right)\right)$$
(ya que $2^n \geq 2^{\log_2(n)}$)

Se demostr'o en el an'alisis uniforme que el orden del tama'no de la entrada es $t = k^2\log_2(m) + \log_2(n)$ (notar que $m_1 = m$).

Sabiendo que
$$2^t = 2^{k^2\log_2(m)+\log_2(n)} = \left(2^{\log_2(m)}\right)^{k^2} 2^{\log_2(n)} = m^{k^2}n$$
y que
$$2^{2^t} = 2^{m^{k^2}n} = \left(2^{n}\right)^{m^{k^2}}$$
se puede ver facilmente que
\begin{itemize}
\item $k^3 \leq 2^{2^t}$ ya que en el segundo, $k$ aparece como exponente y encima al cuadrado.
\item $2^n \leq 2^{2^t}$ ya que en el segundo aparece $2^n$ pero con a'un m'as exponentes.
\item $\log_2(k) \leq 2^{2^t}$ trivial a partir del primer 'item por ser $\log_2(m) \leq k^3$.
\end{itemize}

Como la 'unica variable que se encuentra en el exponente en $k^32^n\log_2(k)$ es $n$, todo 'ese producto nunca superar'a a $2^{2^t}$. Sucede lo mismo con $k^32^n\log_2(m)$, ya que aqu'i aparece $\log_2(m)$ pero 'este ser'a menor porque $m$ aparece en un exponente en $2^{2^t}$. Luego
$$k^32^n\log_2(k) + k^32^n\log_2(m) \leq 2^{2^t} + 2^{2^t} = 2^{2^t+1} \in O(2^{2^t})$$

Por lo que la complejidad logar'itmica se puede expresar como
$$O(2^{2^t})$$
con lo cual, 'esta es \negrita{exponencial} en funci'on del tama'no de la entrada.


\subsubsection{Fuerza Bruta}
El cambio con respecto a la fuerza bruta recae en Potenciar exclusivamente. Aqu'i, para calcular $A^n$ se ir'a multiplicando el resultado actual con $A$ hasta formar $A^n$. M'as exactamente, se realizar'an $n-1$ multiplicaciones. Por lo tanto, la complejidad ser'a O($(n-1)k^3$) = O($nk^3$).



%%%%%%%%%%%%%%%%%%%%%%%%%%%%%%%%%%%%%%%%%%%%%%%%%%%%%%%%%%%%%%%%%%%%%%
% Resultados
%%%%%%%%%%%%%%%%%%%%%%%%%%%%%%%%%%%%%%%%%%%%%%%%%%%%%%%%%%%%%%%%%%%%%%
\clearpage
\subsection{Resultados}
Los gr'aficos a continuaci'on cuentan la cantidad de operaciones de los algoritmos en funci'on de las variables de entrada. Se decidi'o as'i en lugar de hacerlos en funci'on del tama'no de la entrada para que sean sencillos de analizar.

\imagen{img/Tp1Ej3(multiplicar).png}{14}{Cantidad de operaciones b'asicas de Multiplicar en funci'on de las dimensiones de las matrices}
\imagen{img/Tp1Ej3(cant_mul).png}{14}{Cantidad de multiplicaciones realizadas en Potenciar en funci'on del exponente}
\imagen{img/Tp1Ej3(potenciar).png}{14}{Cantidad de operaciones b'asicas de Potenciar en funci'on del exponente y las dimensiones de las matrices}
\imagen{img/Tp1Ej3(potenciar_fb).png}{14}{Cantidad de operaciones b'asicas de Potenciar mediante el algoritmo 'optimo y el de fuerza bruta en funci'on del exponente y las dimensiones de las matrices}
\clearpage


%%%%%%%%%%%%%%%%%%%%%%%%%%%%%%%%%%%%%%%%%%%%%%%%%%%%%%%%%%%%%%%%%%%%%%
% Discusion
%%%%%%%%%%%%%%%%%%%%%%%%%%%%%%%%%%%%%%%%%%%%%%%%%%%%%%%%%%%%%%%%%%%%%%
\subsection{Discusi'on}
El primer gr'afico muestra que la complejidad te'orica y pr'actica son pr'acticamente la misma. Era de esperar un comportamiento as'i ya que el algoritmo es muy simple. Adem'as, la curva pr'actica est'a bien definida por ser el algoritmo O($k^3$) en cualquier caso, no simplemente el "peor" (porque, en realidad, aqu'i no existe peor caso). Las constantes a la hora de sumar los valores son los responsables de ese $5k^2$ adicional en comparaci'on con el caso te'orico (siguen estando en el mismo orden por ser $k^2 < k^3$).

En el segundo se ve la cantidad de multiplicaciones matriciales que realiza Potenciar en funci'on de $n$. Se verifica que en la pr'actica 'esta cantidad es exactamente $2\log_2(n) - 1$ en el peor caso. Es decir, es exactamente como se calcul'o te'oricamente. Notar que siempre el valor siguiente a un peor caso es de los mejores casos (es decir, cuando el exponente es potencia de 2) y que la cantidad de multiplicaciones en 'estos casos es $\log_2(n)+1$ como tambi'en se calcul'o te'oricamente.

El tercer gr'afico se apoya en los dos anteriores. 'Es decir, como las complejidades te'oricas anteriores fueron v'alidas, es de esperar que 'esta tambi'en lo sea ya que, como se ve en el an'alisis de la complejidad, se basa en 'estas dos. Y efectivamente, el orden de la complejidad te'orica es buena cota de la complejidad pr'actica.

Por 'ultimo, el cuarto gr'afico es el an'alisis contra el algoritmo de fuerza bruta. Como era de esperar, al necesitar m'as multiplicaciones el segundo que el primero la cantidad total de operaciones b'asicas realizadas es notablemente mayor a medida que $n$ y $k$ aumentan.
\clearpage


%%%%%%%%%%%%%%%%%%%%%%%%%%%%%%%%%%%%%%%%%%%%%%%%%%%%%%%%%%%%%%%%%%%%%%
% Conclusiones
%%%%%%%%%%%%%%%%%%%%%%%%%%%%%%%%%%%%%%%%%%%%%%%%%%%%%%%%%%%%%%%%%%%%%%
%\subsection{Conclusiones}
%????????????
